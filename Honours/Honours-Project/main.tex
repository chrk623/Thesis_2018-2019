%%%%%%%%%%%%%%%%%%%%%%%%%%%%%%%%%%%%%%%%%
% Masters/Doctoral Thesis 
% LaTeX Template
% Version 2.5 (27/8/17)
%
% This template was downloaded from:
% http://www.LaTeXTemplates.com
%
% Version 2.x major modifications by:
% Vel (vel@latextemplates.com)
%
% This template is based on a template by:
% Steve Gunn (http://users.ecs.soton.ac.uk/srg/softwaretools/document/templates/)
% Sunil Patel (http://www.sunilpatel.co.uk/thesis-template/)
%
% Template license:
% CC BY-NC-SA 3.0 (http://creativecommons.org/licenses/by-nc-sa/3.0/)
%
%%%%%%%%%%%%%%%%%%%%%%%%%%%%%%%%%%%%%%%%%

%----------------------------------------------------------------------------------------
%	PACKAGES AND OTHER DOCUMENT CONFIGURATIONS
%----------------------------------------------------------------------------------------

\documentclass[
11pt, % The default document font size, options: 10pt, 11pt, 12pt
%oneside, % Two side (alternating margins) for binding by default, uncomment to switch to one side
english, % ngerman for German
singlespacing, % Single line spacing, alternatives: onehalfspacing or doublespacing
%draft, % Uncomment to enable draft mode (no pictures, no links, overfull hboxes indicated)
%nolistspacing, % If the document is onehalfspacing or doublespacing, uncomment this to set spacing in lists to single
%liststotoc, % Uncomment to add the list of figures/tables/etc to the table of contents
%toctotoc, % Uncomment to add the main table of contents to the table of contents
%parskip, % Uncomment to add space between paragraphs
%nohyperref, % Uncomment to not load the hyperref package
headsepline, % Uncomment to get a line under the header
%chapterinoneline, % Uncomment to place the chapter title next to the number on one line
%consistentlayout, % Uncomment to change the layout of the declaration, abstract and acknowledgements pages to match the default layout
]{MastersDoctoralThesis} % The class file specifying the document structure
\usepackage{color}
\usepackage[svgnames]{xcolor}
\usepackage{listings}
\usepackage{textcomp}
\usepackage{float}
\usepackage{amsthm}
\definecolor{listinggray}{gray}{0.9}

\definecolor{bggrey}{grey}{0.095}

\lstset{language=R,
    basicstyle=\ttfamily,
    stringstyle=\color{DarkGreen},
    otherkeywords={0,1,2,3,4,5,6,7,8,9},
    morekeywords={TRUE,FALSE},
    deletekeywords={data,frame,length,as,character},
    %keywordstyle=[1]{\color{blue}},
    %morekeywords = [1]{select},
    commentstyle=\color{black},
    backgroundcolor = \color{listinggray},
    showstringspaces = false
}



\usepackage[utf8]{inputenc} % Required for inputting international characters
\usepackage[T1]{fontenc} % Output font encoding for international characters

\usepackage{mathpazo} % Use the Palatino font by default

\usepackage[backend=bibtex,style=authoryear,natbib=true]{biblatex} % Use the bibtex backend with the authoryear citation style (which resembles APA)

\addbibresource{example.bib} % The filename of the bibliography

\usepackage[autostyle=true]{csquotes} % Required to generate language-dependent quotes in the bibliography

\usepackage{hyperref}
\usepackage{tcolorbox}
%----------------------------------------------------------------------------------------
%	MARGIN SETTINGS
%----------------------------------------------------------------------------------------

\geometry{
	paper=a4paper, % Change to letterpaper for US letter
	inner=2.5cm, % Inner margin
	outer=3.8cm, % Outer margin
	bindingoffset=.5cm, % Binding offset
	top=1.5cm, % Top margin
	bottom=1.5cm, % Bottom margin
	%showframe, % Uncomment to show how the type block is set on the page
}

%----------------------------------------------------------------------------------------
%	THESIS INFORMATION
%----------------------------------------------------------------------------------------

\thesistitle{Survey statistics in a database} % Your thesis title, this is used in the title and abstract, print it elsewhere with \ttitle
\supervisor{Professor Thomas \textsc{Lumley}} % Your supervisor's name, this is used in the title page, print it elsewhere with \supname
\examiner{} % Your examiner's name, this is not currently used anywhere in the template, print it elsewhere with \examname
\degree{Bachelor of Science (Honours)} % Your degree name, this is used in the title page and abstract, print it elsewhere with \degreename
\author{Charco \textsc{Hui}} % Your name, this is used in the title page and abstract, print it elsewhere with \authorname
\addresses{} % Your address, this is not currently used anywhere in the template, print it elsewhere with \addressname
\subject{Biological Sciences} % Your subject area, this is not currently used anywhere in the template, print it elsewhere with \subjectname
\keywords{} % Keywords for your thesis, this is not currently used anywhere in the template, print it elsewhere with \keywordnames
\university{\href{http://www.auckland.ac.nz}{The University of Auckland}} % Your university's name and URL, this is used in the title page and abstract, print it elsewhere with \univname
\department{\href{https://www.stat.auckland.ac.nz/en.html}{Department of Statistics}} % Your department's name and URL, this is used in the title page and abstract, print it elsewhere with \deptname
%\group{\href{http://researchgroup.university.com}{Research Group Name}} % Your research group's name and URL, this is used in the title page, print it elsewhere with \groupname
%\faculty{\href{http://faculty.university.com}{Faculty Name}} % Your faculty's name and URL, this is used in the title page and abstract, print it elsewhere with \facname

\AtBeginDocument{
\hypersetup{pdftitle=\ttitle} % Set the PDF's title to your title
\hypersetup{pdfauthor=\authorname} % Set the PDF's author to your name
\hypersetup{pdfkeywords=\keywordnames} % Set the PDF's keywords to your keywords
}
%\graphicspath{{/img}}
\begin{document}

\frontmatter % Use roman page numbering style (i, ii, iii, iv...) for the pre-content pages

\pagestyle{plain} % Default to the plain heading style until the thesis style is called for the body content

%----------------------------------------------------------------------------------------
%	TITLE PAGE
%----------------------------------------------------------------------------------------

\begin{titlepage}
\begin{center}

\vspace*{.06\textheight}
{\scshape\LARGE \univname\par}\vspace{1.5cm} % University name
\textsc{\Large Honours Project}\\[0.5cm] % Thesis type

\HRule \\[0.4cm] % Horizontal line
{\huge \bfseries \ttitle\par}\vspace{0.4cm} % Thesis title
\HRule \\[1.5cm] % Horizontal line
 
\begin{minipage}[t]{0.4\textwidth}
\begin{flushleft} \large
\emph{Author:}\\
\href{https://github.com/chrk623}{\authorname} % Author name - remove the \href bracket to remove the link
\end{flushleft}
\end{minipage}
\begin{minipage}[t]{0.4\textwidth}
\begin{flushright} \large
\emph{Supervisor:} \\
\href{https://www.stat.auckland.ac.nz/people/tlum005}{\supname} % Supervisor name - remove the \href bracket to remove the link  
\end{flushright}
\end{minipage}\\[3cm]
 
\includegraphics[scale=0.1]{Latex/img/uoalogo.png}

\vfill

\large \textit{A thesis submitted in fulfilment of the requirements\\ for the degree of \degreename}\\[0.3cm] % University requirement text
\textit{in the}\\[0.4cm]
\groupname\\\deptname\\[2cm] % Research group name and department name
 
\vfill

{\large June 5, 2017}\\[4cm] % Date
%\includegraphics{Logo} % University/department logo - uncomment to place it
 
\vfill
\end{center}
\end{titlepage}

%----------------------------------------------------------------------------------------
%	ABSTRACT PAGE
%----------------------------------------------------------------------------------------

\begin{abstract}
%\addchaptertocentry{\abstractname} % Add the abstract to the table of contents
%The Thesis Abstract is written here (and usually kept to just this page). The page is kept centered %vertically so can expand into the blank space above the title too\ldots
\addchaptertocentry{\abstractname} % Add the acknowledgements to the table of contents
Multistage surveys can rise to moderately large data sets (tens of millions of rows).  Most current software for survey analysis reads the data into memory, the {\bf survey} package in {\sf R} provides fairly comprehensive analysis features for complex surveys which are small enough to fit into memory easily, however, most of the computations can actually be expressed as database operations. There is already a similar approach with the {\bf sqlsurvey} package in {\sf R} which performs substantial computation in {\sf SQL} in the database, importing only small summary tables into {\sf R}, this approach scales to very large surveys such as the American Community Survey and the Nationwide Emergency Department Sample, but this approach causes compatibility issues with different types of databases. Therefore, in this project I will work on implementing {\sf R} functions and testing some survey computations using the {\bf dplyr} and {\bf dbplyr} {\sf R} package as a efficient and portable database interface.

\end{abstract}

%----------------------------------------------------------------------------------------
%	ACKNOWLEDGEMENTS
%----------------------------------------------------------------------------------------

\begin{acknowledgements}
\addchaptertocentry{\acknowledgementname} % Add the acknowledgements to the table of contents
I would like to acknowledge my supervisor, Professor Thomas Lumley with my deepest appreciation. I would like to thank him for his patience in sharing his expertise, without his help, this project would not be possible. \\

Lastly, i would like to thank my friends and family for their continuous support.
\end{acknowledgements}

%----------------------------------------------------------------------------------------
%	LIST OF CONTENTS/FIGURES/TABLES PAGES
%----------------------------------------------------------------------------------------

\tableofcontents % Prints the main table of contents

\listoffigures % Prints the list of figures

%\listoftables % Prints the list of tables

%----------------------------------------------------------------------------------------
%	ABBREVIATIONS
%----------------------------------------------------------------------------------------

% \begin{abbreviations}{ll} % Include a list of abbreviations (a table of two columns)

% \textbf{LAH} & \textbf{L}ist \textbf{A}bbreviations \textbf{H}ere\\
% \textbf{WSF} & \textbf{W}hat (it) \textbf{S}tands \textbf{F}or\\

% \end{abbreviations}

%----------------------------------------------------------------------------------------
%	PHYSICAL CONSTANTS/OTHER DEFINITIONS
%----------------------------------------------------------------------------------------

% \begin{constants}{lr@{${}={}$}l} % The list of physical constants is a three column table

% % The \SI{}{} command is provided by the siunitx package, see its documentation for instructions on how to use it

% Speed of Light & $c_{0}$ & \SI{2.99792458e8}{\meter\per\second} (exact)\\
% %Constant Name & $Symbol$ & $Constant Value$ with units\\

% \end{constants}

%----------------------------------------------------------------------------------------
%	SYMBOLS
%----------------------------------------------------------------------------------------

% \begin{symbols}{lll} % Include a list of Symbols (a three column table)

% $a$ & distance & \si{\meter} \\
% $P$ & power & \si{\watt} (\si{\joule\per\second}) \\
% %Symbol & Name & Unit \\

% \addlinespace % Gap to separate the Roman symbols from the Greek

% $\omega$ & angular frequency & \si{\radian} \\

% \end{symbols}

%----------------------------------------------------------------------------------------
%	DEDICATION
%----------------------------------------------------------------------------------------

% \dedicatory{For/Dedicated to/To my\ldots} 

%----------------------------------------------------------------------------------------
%	THESIS CONTENT - CHAPTERS
%----------------------------------------------------------------------------------------

\mainmatter % Begin numeric (1,2,3...) page numbering

\pagestyle{thesis} % Return the page headers back to the "thesis" style

% Include the chapters of the thesis as separate files from the Chapters folder
% Uncomment the lines as you write the chapters

% Chapter Template

\chapter{Introduction} % Main chapter title

\label{Introduction} % Change X to a consecutive number; for referencing this chapter elsewhere, use \ref{ChapterX}

%----------------------------------------------------------------------------------------
%	SECTION 1
%----------------------------------------------------------------------------------------

\section{Background}\label{c1.1}

Currently, there is already a {\bf survey} package  \citep{surveypackage} in {\sf R} which is at a stable production status, it provides survey analysis, including graphics, estimation and inference. It also supports both replicate-weight and Taylor linearisation standard errors, and can efficiently handle multistage stratified designs without replacements. However, it requires the data sets to be stored in a data frame in memory. For most survey data sets this is not a problem, however, nowadays there are a number of large survey data sets, for example the American Community Survey (ACS) includes 3,000,000 people per year, and the Nationwide Emergency Department Sample (NEDS) includes more than 25,000,000 hospital visit records per year. \\

In {\sf R}, there are currently two approaches to analyse survey data sets in a database. The first is to use the {\bf survey} package, with its database back-end function, the data sets can be loaded into memory without any problem, but the time to analysis the data may not be promising when the data sets are too large.\\

Another approach is to perform as much computation as possible directly in the database, so that only small bits of data or numbers are transferred into memory when necessary. This approach is more efficient but is less flexible, since mathematics and statistical operations are limited in a database. Another advantage of this approach is that if the database is powerful, then the computation would be faster than just using a standard laptop or desktop. 

The second approach is implemented in the {\bf sqlsurvey} {\sf R} package \citep{sqlsurveypackage}}, however, codes which communicates with the databases are written in "hand-written {\sf SQL} code". Therefore, it would be hard to maintain and would cause compatibility issues between different types of databases. Not only codes may look different, there is also a major inconsistency in evaluating the code, for example dealing with missing values. Despite the attempt of standardising the {\sf SQL} standards between multiple companies, issues of portability still remains.\\

The better approach to analyse survey data sets in a database would be to use the {\sf R} packages {\bf dplyr} \citep{dplyrpackage} and {\bf dbplyr} \citep{dbplyrpackage} as a database interface. Since these packages are maintained by experts at Rstudio, it is likely that these packages are more stable than others, bug fixes and updates would also be quick. Most importantly, it is more portable where its compatibility extends to powerful databases back ends like PostgreSQL and Google BigQuery. So, this project will implement a set of functions to analyse survey data sets with the second approach, named {\bf svydb}, and evaluate its speed on large survey data sets.

\section{Survey data in SQL}
As mentioned in section \ref{c1.1}, when we are analysing large survey data sets, it would be more feasible to do it in a database. Some commonly used survey statistics are survey mean, survey total, summaries and regression.\\

Survey totals, means and summaries can be easily computed in a database, since it only requires simple arithmetic like summing, multiplications, divisions along with some grouping. 

Regression may require a bit more work, since it requires matrix operations which are not supported in {\sf SQL}, however by loading a few chunks of small matrices into memory, it can still be easily implemented in a database, since after all regression coefficients and their variances only requires sums and multiplications.\\

More details of the calculations and difficulties will be discussed later on in Chapter \ref{c2}.

\section{Coding with {\bf dplyr} and {\bf dbplyr}}

\subsection{Introduction to dplyr} \label{c1.2.1}
The {\bf dplyr} package was implemented to manipulate, clean and construct data. With this package, data manipulation and data exploration can be done easily and quickly, since they are written in a computationally efficient manner. 

The package contains a few common data manipulating functions such as selecting specific columns, arranging or creating new columns, filtering rows, merging data (joins) and summarising data by groups. Other features such as simple statistics operations are also included in the package.

\subsection{Pipes}
The pipe operator ({\ttfamily \%>\%}) first appeared in {\bf magrittr} package \citep{magrittrpackage}, and is created to make codes more readable.
The pipe operator inputs the object on the left-hand side of the pipe into the function on right-hand side. Some basic piping are as follows:

\begin{itemize}

\item {\ttfamily x \%>\% f} is equivalent to {\ttfamily f(x).}

\item {\ttfamily x \%>\% f(y)} is equivalent to {\ttfamily f(x, y).} 

\end{itemize}
\\
\vspace{10}
It is rather useful when we have multiple steps while we are transforming data sets, because naturally we read from left to right. For example, with traditional coding, reading is always from inside out,
\begin{lstlisting}
> x = sample(10)
> summary(diff(exp(floor(cos(x)))))
\end{lstlisting}
with piping, it is much easier to read,
\begin{lstlisting}
> x %>% cos() %>% floor() %>% exp() %>% diff() 
    %>% summary() 
\end{lstlisting}
\\
\vspace{10}
Though the pipe has its advantages, there are also times that it is not useful. It would not be useful when the intermediate variables are needed or when the intermediate variables require heavy computation.

\subsection{dplyr's SQL compatibility (dbplyr)}
As mentioned in section \ref{c1.2.1}, there are six basic functions in {\bf dplyr}. These functions are all related to the basic {\sf SQL} queries.

%https://www.listendata.com/2016/08/dplyr-tutorial.html
\begin{center}
\begin{tabular}{ |l|l|l|l| } 
\hline
\textbf{dplyr Function} & \textbf{Description} & \textbf{Equivalent SQL} \\
\hline
select() & Selecting columns (variables) & SELECT \\ 
filter () & Filter (subset) rows. & WHERE \\ 
group\_by() & Group the data & GROUP BY \\ 
arrange() & Sort the data & ORDER BY \\ 
join() & Joining tables & JOIN \\ 
mutate() & Creating New Variables (Columns) & COLUMN ALIAS \\ 
\hline
\end{tabular}
\end{center}

With {\bf dbplyr}, when these {\bf dplyr} functions are applied onto a sql table, they automatically translate itself into {\sf SQL} queries. For example,

\begin{lstlisting}
> mtdb %>% select(mpg, gear) %>% group_by(gear) %>% 
    summarise(sum_mpg = sum(mpg)) %>% head(3) %>% 
    show_query()
 
 # <SQL>
 # SELECT "gear", SUM("mpg") AS "sum_mpg"
 # FROM (SELECT "mpg" AS "mpg", "gear" AS "gear"
 # FROM "mtcars") "gaecowztcc"
 # GROUP BY "gear"
 # LIMIT 3
\end{lstlisting}
With this approach, {\bf dplyr} does not actually do any work, its job is only to translate the codes into {\sf SQL} and gives the database instructions. Another advantage of this method is that the intermediate variables between the pipes only builds up the query and does not get evaluated nor is stored anywhere. 

\subsection{Quasi-quotation}
Programming with {\bf dplyr} relies on a concept called the quasi-quotation, also known as non-standard evaluation, it means that while we are doing some evaluation with {\bf dplyr} in {\sf R}, we are not using {\sf R}'s standard evaluation method. For example, with {\sf R}'s standard method, 

\begin{lstlisting}
test_func = function(x, y){
    x + y
}  
> x1 = 1; x2 = 2
> test_func(x1, x2)
\end{lstlisting}
{\sf R} looks for the variables {\ttfamily x1} and {\ttfamily x2} in the environment, evaluates them and input their values into the function {\ttfamily test\_func}. \\

However, while programming in {\bf dplyr},
\begin{lstlisting}
> mtcars %>% select(mpg)
\end{lstlisting}
The variable {\ttfamily mpg} is a variable in the data set and cannot be found in the environment, it is quoted and evaluated in a non standard way.\\

Though, it might look useful to use this non-standard evaluation method, but it is more difficult to program with, for example while writing a function,

\begin{lstlisting}
test_fun2 = function(data, x){
    data %>% select(x) %>% head(2)
}
> test_fun2(mtcars, mpg)
# Error: `x` must resolve to integer column positions, 
# not a list 
\end{lstlisting}

Since {\bf dplyr} does not evaluate in the standard way, we cannot just pass a variable in like the standard method. We will need to quote it with the {\ttfamily quo()} or {\ttfamily enquo()} function.

\begin{lstlisting}
test_fun2 = function(data, x){
    x = enquo(x)
    data %>% select(!!x) %>% head(2) %>% tbl_df()
}
> test_fun2(mtcars, mpg)
    
#    mpg
#* <dbl>
#1    21
#2    21
\end{lstlisting}
{\ttfamily enquo()} allows us to quote the variable resulting in a quosure where it contains its expression along with an evaluation environment, so we can pass it into the {\ttfamily select()} function, and {\ttfamily !!} (bang bang) allows us to unquote the variable at evaluation. \\

There are also other similar functions to help us overcome the difficulties while programming with  {\bf dplyr}, like {\ttfamily quos()}, {\ttfamily sym()} and {\ttfamily quo\_name()} which are used in different situations.

\subsection{Common issues while coding with dplyr in a database}

\begin{itemize}
    \item No factor types in {\sf SQL}.
    \item Difficult to code with quasi-quotation.
    \item Cannot do row-wise operations due to the lazy interface. That is, the data sets within a database in {\sf R} will not be loaded into memory unless required.
    \item No matrix operations.
    \item No base {\sf R} functions. 
    \item No distributions.
    \item Inconsistent availability of functions between databases.
\end{itemize}

%\newpage

\section{Layout}

In chapter \ref{c2}, estimation methods and functions will be discussed, and in chapter \ref{c3}, graphics. 

In chapter \ref{c4}, speed of database-based and memory-based implementations will be compared,  chapter \ref{c5} discuss the usability of the functions, and lastly, a discussion in chapter \ref{c6}.
%\include{Chapters/Chapter1}
% Chapter 2

\chapter{Methodology} \label{c2} % Main chapter title

%\label{Chapter1} % For referencing the chapter elsewhere, use \ref{Chapter1} 

%----------------------------------------------------------------------------------------

% Define some commands to keep the formatting separated from the content 
\newcommand{\keyword}[1]{\textbf{#1}}
\newcommand{\tabhead}[1]{\textbf{#1}}
\newcommand{\code}[1]{\texttt{#1}}
\newcommand{\file}[1]{\texttt{\bfseries#1}}
\newcommand{\option}[1]{\texttt{\itshape#1}}

%----------------------------------------------------------------------------------------
\section{Survey Design} \label{c2.1}
When analysing survey data sets in the {\bf survey} package, a survey design ({\bf svydesign()}) is always required. The survey design object combines the data set and all the survey design information needed to analyse it. These objects are used by the survey modelling and summary functions.\\

The set of functions that {\bf surveydb} provides also adapted this concept but with a few modifications, it uses the {\bf R6} \citep{R6package} class system which is encapsulated object orientation programming and is different to the standard {\bf S3} and {\bf S4} in base {\sf R} which uses functional object orientation programming. The main difference between the two is that {\bf S3} and {\bf S4} methods and objects are separate and in {\bf R6}, object contains methods and data. \\

The advantage of encapsulated object orientation programming is that information within the objects are not computed unless it is needed, for example when the sum of all sampling weights are needed, we can compute it by using a method within the object and the value also updates when it's been called on a subset of the object.

\begin{lstlisting}
> nh.dbsurv = svydbdesign(st = SDMVSTRA, wt = WTMEC2YR, 
    id = SDMVPSU, data = nhdb)
> nh.dbsurv$getwt()
[1] 306590681

> nh.dbsurv$subset(Race3 == 3)$getwt()
[1] 192721267
\end{lstlisting}

Therefore, it is much more time efficient than creating a survey design that contains all the information, since the user may not need every information within the object. \\

% Another advantage of this approach is that there will not be any extra generic methods that can be called directly by the user, because these methods are designed specifically for a type of object and usually it should not be called directly by the user.


The {\ttfamily svydbdesign()} function, has four basic arguments,
\begin{center}
    {\ttfamily svydbdesign(st = NULL, id = NULL, wt, data)}
\end{center}

\begin{itemize}
\item $st$ = Column name specifying the strata column. $NULL$ for no strata. 

\item $id$ = Column name specifying the cluster column. $NULL$ for no cluster. 

\item $wt$ = Column name specifying the sampling weights column.

\item $data$ = A data frame or sql table of the survey data set.
\end{itemize}

\\

When the {\ttfamily svydb.design} object is called directly, a brief description of the design will be displayed.

\begin{lstlisting}
> nh.dbsurv
svydb.design, 9756 observation(s), 31 Clusters
\end{lstlisting}

\\

Functions within the {\bf svydb.design} object for users to use are,

\begin{lstlisting}
Classes 'svydb.design', 'R6' <svydb.design>
.
.
clone: function (deep = FALSE) 
getmh: function () 
getwt: function () 
subset: function (..., logical = T) 
subset_rows: function (from, to) 
\end{lstlisting}


\begin{itemize}
\item $clone()$ = Create a clone of the object.

\item $getmh()$ = A table indicating strata and cluster information.

\item $getwt()$ = Compute the sum of the sampling weights.

\item $subset()$ = Subset the design by conditions, similar to $base::subset$. \\
                    e.g.({\ttfamilyd design\$subset(Race3 == 3)} 

\item $subset\_rows()$ = Subset the design by rows. 

e.g.({\ttfamily design\$subset\_rows(1, 100)} )

\end{itemize}

%----------------------------------------------------------------------------------------
\newpage

\section{Population Total} \label{c2:tot} \label{c2.2}
%http://essedunet.nsd.uib.no/cms/topics/weight/3/1.html



The function {\ttfamily svydbtotal()} was designed to estimate the population total in {\sf R} by using {\bf dplyr}, it is compatible with data frames and sql tables and was designed to do as much computation as possible in a database.\\

In the function {\ttfamily svydbtotal()}, the total is computed by using the Horvitz-Thompson estimator \citep{hte}, it is an unbiased estimator of the population total.

$$ \hat{Total} = \sum_{h = 1}^{L} \sum_{i = 1}^{m_h} z_{hi}$$

where,

$$z_{hi} = \sum_{j \in PSU} w_{hij} x_{hij}$$

\begin{itemize}
\item $L$ = number of stratum

\item $m_h$ = number of clusters in stratum $h$

\item $w_{hij}$ = the sample weight in stratum {\emph h} and cluster {\emph i} for observation {\emph j}.
\end{itemize}
\\
Variance estimation of the total uses the Horvitz-Thompson estimator on an influence function.

$$Var(\hat{Total}) =  \sum_{h = 1}^{L} \frac{m_h}{m_h - 1} 
                    \sum_{i = 1}^{m_h} (z_{hi} - \bar{z}_h)^T (z_{hi} - \bar{z}_h)$$

where,
$$\bar{z}_h = \frac{1}{m_h} \sum_{i = 1}^{m_h} z_{hi}$$
\\
\subsection{Usage}
\begin{center}
    {\ttfamily svydbtotal(x, num = T, return.total = F, design)}
\end{center}
\subsection{Arguments}
\begin{itemize}
\item $x$ = Name indicating the variable.

\item $num$ = {\ttfamily TRUE} or {\ttfamily FALSE} indicating whether x is numeric or categorical.

\item $return.total$ = {\ttfamily TRUE} to return only totals, no standard errors.

\item $design$ = svydb.design object.
\end{itemize}

\newpage
\subsection{Examples}
\begin{lstlisting}
> nh.dbsurv = svydbdesign(st = SDMVSTRA, wt = WTMEC2YR, 
                id = SDMVPSU, data = nhdb)
> svydbtotal(x = Race3, design = nh.dbsurv, num = T)
#           Total       SE
# Race3 959380842 61432595

> svydbtotal(x = Race3, design = nh.dbsurv, num = F)
#             Total       SE
# Race3_1  29812316  6112527
# Race3_2  21416164  4485865
# Race3_3 192721267 23431296
# Race3_4  38131538  5561161
# Race3_6  15519529  2367723
# Race3_7   8989867  1468813

> svydbtotal(x = Race3, design = nh.dbsurv , num = T, 
    return.total = T)
#           Total
# Race3 959380842
\end{lstlisting}
\\

Generic functions like {\ttfamily coef()} and {\ttfamily SE()} were also implemented to extract the coefficients and standard errors from a {\bf svydbstat} object.
\begin{lstlisting}
> class(svydbtotal(x = Race3, design = nh.dbsurv, num = T))
# [1] "svydbstat"

> coef(svydbtotal(x = Race3, design = nh.dbsurv , num = T))
# [1] 959380842

> SE(svydbtotal(x = Race3, design = nh.dbsurv , num = T))
#    Race3 
# 61432595 
\end{lstlisting}
\\
\subsection{Difficulties}
\begin{enumerate}
\item In {\sf SQL} if a column contains only numbers, it is not possible to identify whether a column type is factor or numeric, therefore in the {\ttfamily svydbtotal()} function, the user needs to specify it. {\ttfamily num = TRUE}: Numeric, {\ttfamily num = FALSE}: Factor. \label{tot:d1}

\item To compute the population total for a categorical variable, dummy variables are needed. In {\sf SQL}, there are two ways to do this. The first way would be to create new columns for every levels of the variable manually and apply {\ttfamily ifelse/CASE WHEN} to each of the columns. Another way would be to create a small contrast table in memory and use {\ttfamily INNER JOIN} to join the small table onto the data set. The second approach is much faster, and the function {\ttfamily dummy\_mut()} adapts that approach. It creates (mutate) new dummy columns on the right side of the data set. \label{tot:d2}

\item Calculating the variance/standard error is the most complex part of {\ttfamily svydbtotal()}. Therefore the {\ttfamily svyVar()} function is written to calculate the variance of a variable. If the variable is a categorical variable with multiple levels, the calculations will be replicated with {\ttfamily sapply()}. \label{tot:d3}
\end{enumerate}

%----------------------------------------------------------------------------------------
\newpage
\section{Population Mean}
The function {\ttfamily svydbmean()} was designed to estimate the population mean in {\sf R} by using {\bf dplyr}, it is compatible with data frames and sql tables and was designed to do as much computation as possible in a database.\\

In the function {\ttfamily svydbmean()}, the mean is computed by using a ratio estimator rather than the  Horvitz-Thompson estimator. This is a standard in survey software as $N$ may not be known.

$$ \hat{Mean} = \frac{1}{\hat{N}} \sum_{h = 1}^{L} \sum_{i = 1}^{m_h} z_{hi}$$

where,

$$\hat{N} = \sum_{j \in PSU} w_j \text{,\quad} z_{hi} = \sum_{j \in PSU} w_{hij} x_{hij}$$
\\
\begin{itemize}
\item $L$ = number of stratum

\item $m_h$ = number of clusters in stratum $h$

\item $w_{hij}$ = the sample weight in stratum {\emph h} and cluster {\emph i} for observation {\emph j}.
\end{itemize}
\\
Variance estimation of the mean uses the Horvitz-Thompson estimator on an influence function.

$$Var(\hat{Mean}) =  \sum_{h = 1}^{L} \frac{m_h}{m_h - 1} 
                    \sum_{i = 1}^{m_h} (d_{hi} - \bar{d}_h)^T (d_{hi} - \bar{d}_h)$$

where,
$$d_{hi} = \frac{1}{\hat{N}} \sum_{j \in PSU} w_{hij}(x_{hij} - \bar{x}) 
\text{,\quad} 
\bar{d}_h = \frac{1}{m_h} \sum_{i = 1}^{m_h} d_{hi}$$
\\

\subsection{Usage}
\begin{center}
    {\ttfamily svydbmean(x, num = T, return.mean = F, design)}
\end{center}
\subsection{Arguments}
\begin{itemize}
\item $x$ = Name indicating the variable.

\item $num$ = {\ttfamily TRUE} or {\ttfamily FALSE} indicating whether x is numeric or categorical.

\item $return.mean$ = {\ttfamily TRUE} to return only means, no standard errors.

\item $design$ = svydb.design object.
\end{itemize}

\newpage
\subsection{Examples}
\begin{lstlisting}
> nh.dbsurv = svydbdesign(st = SDMVSTRA, wt = WTMEC2YR, 
                id = SDMVPSU, data = nhdb)
> svydbmean(x = Race3, design = nh.dbsurv , num = T)
#         Mean     SE
# Race3 3.1292 0.0674

> svydbmean(x = Race3, design = nh.dbsurv, num = F)
#             Mean     SE
# Race3_1 0.097238 0.0208
# Race3_2 0.069853 0.0154
# Race3_3 0.628595 0.0407
# Race3_4 0.124373 0.0239
# Race3_6 0.050620 0.0080
# Race3_7 0.029322 0.0045
\end{lstlisting}
\\

Generic functions like {\ttfamily coef()} and {\ttfamily SE()} were also implemented to extract the coefficients and standard errors from a {\bf svydbstat} object. This is useful since the {\bf svydbstat} objects are rounded when they are printed, by using {\ttfamily coef()} and {\ttfamily SE()}, the unrounded value can be extracted.
\begin{lstlisting}
> class(svydbmean(x = Race3, design = nh.dbsurv, num = T))
# [1] "svydbstat"

> coef(svydbmean(x = Race3, design = nh.dbsurv, num = T))
# [1] 3.129191

> SE(svydbmean(x = Race3, design = nh.dbsurv, num = T))
#      Race3 
# 0.06735437
\end{lstlisting}
\\
\subsection{Difficulties}
\begin{enumerate}
\item In {\sf SQL} cannot recognise factor variables. Explained in difficulty \ref{tot:d1} from chapter \ref{c2.2} (\hyperref[c2:tot]{Population Total}).

\item {\sf SQL} does not have built-in support for dummy variables. Explained in difficulty \ref{tot:d2} from chapter \ref{c2.2} (\hyperref[c2:tot]{Population Total}).

\item Difficult to compute the variances. Explained in difficulty \ref{tot:d3} from chapter \ref{c2.2} (\hyperref[c2:tot]{Population Total}).
\end{enumerate}

%----------------------------------------------------------------------------------------

\newpage
\section{Regression}
The function {\ttfamily svydblm()} was designed to fit a linear model to survey data in {\sf R} by using {\bf dplyr}, it is compatible with data frames and sql tables and was designed to do as much computation as possible in a database.\\

In the function {\ttfamily svydblm()}, the coefficients are computed by,
$$\hat{\beta} = (X^TWX)^{-1}(X^TWY)$$
\begin{itemize}
\item $W$ = Sampling weights
\end{itemize}
\\
Variance estimation of the coefficients uses a similar approach as survey mean/total,
$$Var_{pq}(\hat{\hat{\beta}}) =  \sum_{h = 1}^{L} \frac{m_h}{m_h - 1} 
                    \sum_{i = 1}^{m_h} (z_{hip} - \bar{z}_{hp})^T (z_{hiq} - \bar{z}_{hq})$$
where,
$$z_{hi} = x_{hij} w_{hij} (y_{hij} - \mu_{hij}) 
\text{,\quad} 
\bar{z}_h = \frac{1}{m_h} \sum_{i = 1}^{m_h} z_{hi}$$
\\
And by using the sanwich estimator,
$$cov(\hat{\beta}) =  (X^TWX)^{-1}Var_{pq}(\hat{\beta})(X^TWX)^{-1}$$

\begin{itemize}
\item $L$ = number of stratum

\item $m_h$ = number of clusters in stratum $h$

\item $w_{hij}$ = the sample weight in stratum {\emph h} and cluster {\emph i} for observation {\emph j}.

\item $p,q$ = Indicator function for variables $p$ and $q$.
\end{itemize}

\subsection{Usage}
\begin{center}
    {\ttfamily svydblm(formula, design)}
\end{center}
\subsection{Arguments}
\begin{itemize}
\item $formula$ = Model formula.

\item $design$ = svydb.design object.
\end{itemize}

\newpage
\subsection{Examples}
\begin{lstlisting}
> nh.dbsurv = svydbdesign(st = SDMVSTRA, wt = WTMEC2YR, 
                        id = SDMVPSU, data = nhdb)
> svydblm(DirectChol ~ Age + BMI + factor(Gender), 
    design = nh.dbsurv)
# svydb.design, 9756 observation(s), 31 Clusters
#
# Survey design:
# svydbdesign(st = SDMVSTRA, id = SDMVPSU, wt = WTMEC2YR, 
#    data = nhdb)
#
# Call:
# svydblm(formula = DirectChol ~ Age + BMI + factor(Gender), 
#    design = nh.dbsurv)
#
# Coefficients:
#       intercept  Age        BMI        Gender_2 
# [1,]   1.632111   0.003254  -0.018636   0.218086
\end{lstlisting}

\\
The {\ttfamily summary()} function to obtain the summary of the model and {\ttfamily predict()} function to predict using new data-sets with the model was also implemented,
\\
\begin{lstlisting}
# dbfit = svydblm(formula = DirectChol ~ Age + BMI, 
#   design = nh.dbsurv)

> summary(dbfit)
# Call:
# svydblm(formula = DirectChol ~ Age + BMI,
#   design = nh.dbsurv)
#
# Survey design:
# svydbdesign(st = SDMVSTRA, id = SDMVPSU, wt = WTMEC2YR, 
#   data = nhdb)
#
# Coefficients:
#             Estimate Std. Error t value Pr(>|t|)    
# intercept  1.7263135  0.0279703  61.719  < 2e-16 ***
# Age        0.0034161  0.0003428   9.966 5.23e-08 ***
# BMI       -0.0182468  0.0011277 -16.181 6.63e-11 ***
# ---
# Signif. codes:  0 ‘***’ 0.001 ‘**’ 0.01 ‘*’ 
#   0.05 ‘.’ 0.1 ‘ ’ 1

> predict(dbfit, newdata = data.frame(Age = 1:3, BMI = 4:6))
#     link     SE
# 1 1.6567 0.0242
# 2 1.6419 0.0233
# 3 1.6271 0.0225
\end{lstlisting}
\\\newpage
Other generic function includes {\ttfamily coef()}, {\ttfamily SE()}, {\ttfamily vcov()} and {\ttfamily residuals()}.
\\
\begin{lstlisting}
# dbfit = svydblm(formula = DirectChol ~ Age + BMI, 
#   design = nh.dbsurv)

> coef(dbfit)
#      intercept         Age         BMI
# [1,]  1.726314 0.003416124 -0.01824676

> SE(dbfit)
#    intercept          Age          BMI 
# 0.0279703171 0.0003427746 0.0011276877

> vcov(dbfit)
#               intercept           Age           BMI
# intercept  7.823386e-04  3.769663e-06 -2.806340e-05
# Age        3.769663e-06  1.174944e-07 -2.496403e-07
# BMI       -2.806340e-05 -2.496403e-07  1.271680e-06

> head(residuals(dbfit), 3) 
# Source:   lazy query [?? x 1]
# Database: MonetDBEmbeddedConnection
#   residuals
#       <dbl>
# 1    -0.316
# 2    -0.318
# 3    -0.733
\end{lstlisting}



\\
\subsection{Difficulties}
\begin{enumerate}
\item {\sf SQL} does not have built-in support for dummy variables. Explained in difficulty \ref{tot:d2} from chapter \ref{c2.2} (\hyperref[c2:tot]{Population Total}).

\item In {\sf SQL} matrix multiplications are not supported, however, it can still be implemented since matrix multiplications only requires addition and multiplication. For example with matrix $X$, by calculating the sums of products of the first column of the matrix with the rest of the columns (including the first column), we get the first row of the $X^TX$ matrix. To get the whole $X^TX$ matrix we only need to repeat the process with different columns. However, the inverse of a matrix is not possible in {\sf SQL}, so to compute $(X^TX)^{-1}$, we need to pull the matrix into memory. 

\item The variance for the regression coefficients are a bit more complicated than computing the variance for mean/total, since for mean/total we only need the diagonal of the co-variance matrix. However, the variances of the regression coefficients requires the whole covariance matrix. This means that more replication with different combinations of $z_{hip}$/$z_{hiq}$ will be needed, but we only need the upper triangle or the lower triangle of the matrix, since it is a symmetric matrix.
\end{enumerate}



%----------------------------------------------------------------------------------------
\newpage
\section{Quantiles}

The function {\ttfamily svydbquantile()} was designed to compute the medians/quantiles from survey data in {\sf R} by using {\bf dplyr}, it is compatible with data frames and a {\bf few types} of sql tables and was designed to do as much computation as possible in a database.\\

To estimate the median/quantiles, a standard probabilistic algorithm is used. For example to estimate the median,
\begin{enumerate}
    \item \label{qe1} Take a sample of size $n^{2/3}$ from the data set and read it into memory. (Proof below)

    \item \label{qe2} Compute the 99\% confidence interval $[a,b]$ of the 0.5 quantile by using {\ttfamily svyquntile()} from the survey package.
    
    \item \label{qe3} Read in the data set where the observations are between $a$ and $b$.
    
    \item \label{qe4} Sort the data and compute the cumulative sum of the weights of the read in observations, $w_{n} = \sum_{i = 1}^{n} w_i$
    
    \item \label{qe5} Find out if the median is within the read in data set. Since $median = 0.5 \times W$, we can find out by searching if median equals or is between $w_{n}$.
    
    \item \label{qe6} If the median is not found, repeat.
\end{enumerate}
\\
Other quantiles are computed with the same method. \\

        
Though by using this method to compute the survey quantile requires at least two sets of data to be read into memory, however it is still much more efficient than sorting and calculating the cumulative sum for the whole data set, since $n^{2/3}$ is relatively small compared to whole data set (One million observations, $10000000^{2/3} = 10000$).

\hspace{1}

\begin{tcolorbox}
%\begin{proof}
   Let $M$ be a random sample of $N$.}\\
   
   The first set of data read into memory has $M$ points and its confidence interval length is $\propto M^{-1/2}$. (step \ref{qe1}) \\
   
   Number of points in the second read is $\beta N M^{-1/2}$. (step \ref{qe3}) \\
   
   Total number of points read in is $M + \beta N M^{-1/2}$. Minimise,
   
   $$\frac{\partial}{\partial M} = 1 + N \left(\frac{-1}{2} M^{-3/2} \right) = 0$$
   
   $$1 = \frac{1}{2} N M^{-3/2}$$
   
   $$2M^{-3/2} = N$$
   
   $$M \propto N^{2/3}$$
%\end{proof}
\end{tcolorbox}



\newpage

\subsection{Usage}
\begin{center}
    {\ttfamily svydbquantile(x, quantiles = 0.5, design)}
\end{center}
\subsection{Arguments}
\begin{itemize}
\item $x$ = Name indicating the variable.

\item $quantiles$ = Quantiles to estimate, a number, or a vector of numbers for multiple quantiles. Default to 0.5.

\item $design$ = svydb.design object.
\end{itemize}

\subsection{Examples}
\begin{lstlisting}
> db.dbsurv = svydbdesign(st = SDMVSTRA, wt = WTMEC2YR, 
                        id = SDMVPSU, data = nhdb)
> svydbquantile(x = Age, quantile = 0.5, design = nh.dbsurv)
# 0.5 
#  37 
 
> svydbquantile(x = BMI, quantile = c(0.25,0.75), 
                            design = nh.dbsurv)
# 0.25 0.75 
# 21.7 30.6 
\end{lstlisting}
\\
\subsection{Difficulties}
\begin{enumerate}
\item To compute the survey quantile, a sample of the data-set is needed. However, different types of {\sf SQL} uses different queries for sampling tables and the {\ttfamily sample\_frac()} function from {\bf dplyr} currently only works with {\bf Spark} \citep{sparkpackage} database connections and local dataframes/tibbles. Sampling in {\bf MonetDB} \citep{monetdb} was also implemented. Therefore currently, the {\ttfamily svydbquantile()} is only tested with local dataframes and database connections that are mentioned above. It may be possible that it is compatible with other types of connections but they are to be tested. Another possibility is that the {\ttfamily sample\_frac()} function will be extended in the future to support more database connections.

\item Since to compute the survey quantile in {\ttfamily svydbquantile()}, the inputted data-set will be sized down into $n^{2/3}$, so if the inputted data-set is a pre-subsetted data-set then it means that the data-set will be subsetted at least twice to compute the quantiles. This could be a problem because {\ttfamily svydbquantile()} runs through {\ttfamily svyquantile()} form the {\bf survey} package which uses {\ttfamily svymean()} within it. If the data-set is too small it may lose enough information and may cause mathematical errors. For example, if there are only one cluster within a strata then it will cause the equation to divide by zero. 
\\
To overcome this, there is a option within the {\bf survey} package called "survey.loney.psu", if we set this option to "adjust", e.g. {\ttfamily options("survey.lonely.psu" = "adjust")}, it will allow survey statistic computations even if there is only one cluster within a strata.

\end{enumerate}

%----------------------------------------------------------------------------------------

%----------------------------------------------------------------------------------------
\newpage
\section{Survey Tables}
The function {\ttfamily svydbtable()} was designed to create contingency tables for survey data, it is compatible with data frames and sql tables and was designed to do as much computation as possible in a database.\\

Each cell within the table is computed with the same method as {\ttfamily svydbtotal()}.

\subsection{Usage}
\begin{center}
    {\ttfamily svydbtable(formula, design, as.local = F)}
\end{center}
\subsection{Arguments}
\begin{itemize}
\item $formula$ = A formula specifying margins for the table, only + can be used.

\item $design$ = svydb.design object.

\item $as.local$ = A logical value indicating the returning object type. Default is database tables, {\ttfamily tbl\_sql}.
\end{itemize}

\subsection{Examples}

\begin{lstlisting}
> nh.dbsurv = svydbdesign(st = SDMVSTRA, wt = WTMEC2YR, 
                        id = SDMVPSU, data = nhdb)

> svydbtable(~MaritalStatus, design = nh.dbsurv)
# Source:     lazy query [?? x 2]
# Database:   MonetDBEmbeddedConnection
# Ordered by: MaritalStatus
#   MaritalStatus         wt
#           <int>      <dbl>
# 1             1 118752657.
# 2             2  12600347.
# 3             3  23868539.
# 4             4   5486968.
# 5             5  44543092.
# 6             6  18664186.                       

> svydbtable(~MaritalStatus, design = nh.dbsurv, 
    as.local = T)
#   MaritalStatus         wt
#           <int>      <dbl>
# 1             1 118752657.
# 2             2  12600347.
# 3             3  23868539.
# 4             4   5486968.
# 5             5  44543092.
# 6             6  18664186.
\end{lstlisting}
\newpage
\begin{lstlisting}
> svydbtable(~Race3 + Smoke100, design = nh.dbsurv, 
    as.local = T)
# A tibble: 6 x 5
#   Race3 Smoke100_1 Smoke100_2 Smoke100_7 Smoke100_9
#   <int>      <dbl>      <dbl>      <dbl>      <dbl>
# 1     1   6177437.  11124548.         0          0 
# 2     2   5375656.   9313414.         0          0 
# 3     3  71234238.  77533263.         0      29247.
# 4     4   9687657.  16031251.     22142.     14686.
# 5     6   3019921.   8645276.         0      14966.
# 6     7   3070772.   2650765.         0          0 


> svydbtable(~Race3 + Work + Gender, design = nh.dbsurv, 
    as.local = T)

# $`Gender  =  1`
# A tibble: 6 x 6
#   Race3    Work_1   Work_2   Work_3    Work_4 Work_7
#   <int>     <dbl>    <dbl>    <dbl>     <dbl>  <dbl>
# 1     1  7063412.  118589.  805525.  2148047. 54555.
# 2     2  5066507.   50966.  283436.  2056117.     0 
# 3     3 48916000. 1670881. 3052152. 22756262.     0 
# 4     4  6557730.  166729.  812680.  5201428.     0 
# 5     6  3689443.   79793.  191406.  1764683.     0 
# 6     7  1747070.   34664.  258597.  1182825.     0 

# $`Gender  =  2`
# A tibble: 6 x 6
#   Race3    Work_1   Work_2   Work_3    Work_4 Work_7
#   <int>     <dbl>    <dbl>    <dbl>     <dbl>  <dbl>
# 1     1  4839422.   80842.  273991.  4040679.      0
# 2     2  4052770.   20057.  400463.  3907392.      0
# 3     3 42775016. 1370755. 2995602. 34142499.      0
# 4     4  7486159.  280862.  915095.  7070385.      0
# 5     6  3557006.  111248.  209497.  2803386.      0
# 6     7  1810965.   17984.  114653.  1255186.      0

\end{lstlisting}
\\

%----------------------------------------------------------------------------------------
\newpage
\section{Survey Statistic on Subsets}
The function {\ttfamily svydbby()} was designed to compute survey statistics on subsets of the data in {\sf R} by using {\bf dplyr}, it is compatible with data frames and sql tables and was designed to do as much computation as possible in a database.\\

This function creates a number of subsets based on the conditions given by the user and computes the desired survey statistic on all the subsets. Currently, it is only compatible with {\ttfamily svydbtotal()} and {\ttfamily svydbmean()}.\\

\subsection{Usage}
\begin{center}
    {\ttfamily svydbby(x, by, FUN, design, ...)}
\end{center}

\subsection{Arguments}
\begin{itemize}
\item $x$ = A variable specifying the variable to pass to FUN.

\item $by$ = A variable specifying factors that define the subsets.

\item $FUN$ = A function indicating the desired survey statistics.

\item $design$ = svydb.design object.

\item $...$ = Other arguments to pass to FUN.
\end{itemize}

\subsection{Examples}
\begin{lstlisting}
> nh.dbsurv = svydbdesign(st = SDMVSTRA, wt = WTMEC2YR, 
                        id = SDMVPSU, data = nhdb)
> svydbby(x = Age, by = Gender, FUN = svydbmean, 
                design = nh.dbsurv, num = T)
# $Age
#                 Mean        SE
# Gender == 1 36.21035 0.8387459
# Gender == 2 38.09899 0.6721789

> svydbby(x = BMI, by = Race3, FUN = svydbtotal, 
                design = nh.dbsurv, num = T)
# $BMI
#                 Total        SE
# Race3 == 3 5004822100 612188966
# Race3 == 1  737939057 153551662
# Race3 == 6  347435902  52876464
# Race3 == 4 1020476221 154630793
# Race3 == 7  227729724  39637006
# Race3 == 2  550207484 118331941
\end{lstlisting}
\\

%----------------------------------------------------------------------------------------

\newpage
\section{Replicate Weights}
In survey data sets, standard errors can never be known with any certainty and are only estimated. Replicate weights lets us to use a single sample with different sampling weights to capture the characteristics of multiple samples, it allows us to compute more informed standard error estimates, this method is similar to bootstrap and jackknife sampling. Though computing standard errors from replicate weights usually result in them getting bigger, but the increase usually is not large enough that it can alter the significance level. 

Another reason for us to use replicate weights is that it provides a less complex way to compute the standard errors. \\

Currently, replicate weights are available in a number of data sets, for example the American Community Survey and Puerto Rican Community Survey data. In these data sets, there are 80 separate replicate weights at the household/person level which allows us to compute the standard errors.

\subsection{Replicate Standard Errors}
To compute the replicate standard errors, there are three steps.
\begin{enumerate}
    \item Compute the survey statistics of interest with the full sample weights.
    
    \item Rerun the analysis with each set of the replicate weights.
    
    \item Calculate the standard error,
    
            $$ SE(X) = \sqrt{ s \sum_{r = 1}^{n(r)} (X - X_r)^2 }$$
                
            \begin{itemize}
                \item $X$ = Result of the survey statistics using the full sample weights.

                \item $X_r$ = Result of the survey statistics using the r'th set of the replicate weights. 

                \item $s$ = Scale multiplier. i.e. $\frac{4}{80}$ for the American Community Survey.
            \end{itemize}

\end{enumerate}

\subsection{Survey Replicate Design}

Similarly, there is a survey replicate design like the survey design from section \ref{c2.1}.\\

Currently, replicate statistics that are supported are survey totals and means. Therefore, there is no need to provide the stratification and clustering information to {\ttfamily svydbrepdesign()}.

\begin{center}
    {\ttfamily svydbrepdesign(wt, repwt, scale, data)}
\end{center}

\begin{itemize}
\item $wt$ = Column name specifying the sampling weights column.

\item $repwt$ = A regular expression that matches the names of the replication weight variables.

\item $data$ = A data frame or sql table of the survey data set.
\end{itemize}

\subsection{Examples}
\begin{lstlisting}
> hde.repsurv = svydbrepdesign(wt = WGTP, repwt="wgtp[0-9]+", 
      scale = 4/80, data = ss16hde)
> hde.repsurv
# svydb.repdesign, 4582 observation(s), 
#   80 sets of replicate weights, scale = 0.05
\end{lstlisting}

\subsection{Population Total with replicate weights}

Arguments are the same as {\ttfamily svydbtotal()}, but with an extra argument. \\
\begin{itemize}
    \item $return.replicate$ = $TRUE$ to return the replicate statistics.
\end{itemize}

\begin{lstlisting}
> hde.dbrepsurv = svydbrepdesign(wt = WGTP, 
    repwt = "wgtp[0-9]+", scale = 4/80, data = ss16hde)
> svydbreptotal(x = BATH, design = hde.dbrepsurv , num = T)
#       Total     SE
# BATH 429410 755.04

> svydbreptotal(x = FS, design = hde.dbrepsurv , num = F)
#       Total     SE
# FS_1  37392 2151.8
# FS_2 313694 3033.6

> (svydbreptotal(x = HHT, design = hde.dbrepsurv , 
        num = T, return.replicates = T)$replicates)[,1:3]
#    wgtp1  wgtp2  wgtp3
#    <dbl>  <dbl>  <dbl>
# 1 975397 973284 987185

> coef(svydbreptotal(x = BATH, design = hde.dbrepsurv,
        num = T))
# [1] 429410      

> SE(svydbreptotal(x = BATH, design = hde.dbrepsurv, 
    num = T))
#     BATH  
# 755.0406    
\end{lstlisting}
\subsection{Population Mean with replicate weights}
All arguments are the same as {\ttfamily svydbreptotal()}.
\begin{lstlisting}
> svydbrepmean(x = BATH, design = hde.dbrepsurv , num = T)
#        Mean     SE
# BATH 1.0076 0.0018

> svydbrepmean(x = FS, design = hde.dbrepsurv , num = F)
#        Mean     SE
# FS_1 0.1065 0.0061
# FS_2 0.8935 0.0061
\end{lstlisting} 
% Chapter Template
\graphicspath{{/img}}
\chapter{Out-of-memory Graphics} \label{c3} % Main chapter title

Graphs are useful in statistics because it allows us to have a visual perspective of the data sets or to present them in a more meaningful way. However, when implementing graphics with survey data sets, it may become more difficult because the data sets are often large. For example, if every point of the large data set is plotted in a scatter plot, there will be too many points in the graph which will overlap and hide each other. There are two ways to over come this problem,

\begin{enumerate}
    \item Use different symbols, colour or sizes to represent different points or group of points. This is only suitable for data sets that are moderately large. If the data sets are too large, the points will still overlap each other.
    
    \item Condense multiple points into one point with an algorithm and use colours to represent the density of the sampling weights within the point. This will be discussed with more detail in section \ref{c3.3}.
\end{enumerate}

Another approach would be to take a sample from the full data set based on the estimated population distribution and plot the sample. \\

Currently, there are very few {\sf R} packages which allows us to plot graphs with data sets which are stored in a database. One of them is {\bf dbplot} \citep{dbplotpackage}, it is simple and light-weighted but it is in its early development stage. As for survey data sets, except for the {\bf sqlsurvey} package mentioned in section \ref{c1.1}, there are no packages that will allows us to plot data sets with sampling weights without having the data sets stored in memory.\\

Like the {\bf survey} package, {\bf svydb} also provides a set of tools for plotting survey data sets and is implemented with {\bf ggplot2} \citep{ggplotpackage}, but with less flexibility and options. Similar to the previous chapters, every graphical functions within {\bf svydb} is designed to do as much computation in the database as possible.

For some type of graphs, it is reasonable to produce them with data sets outside of memory, since to plot them, we don't need the whole data set in memory (computing the statistics for the graphs within the database may even be faster than transferring the whole data set into memory). For example, plotting a histogram with one variable, the only information that are needed to plot it is the location of the breaks for the x-axis and the density of the corresponding break for the y-axis. Say if there are 30 breaks, then at most we will only need 60 numbers in memory, 30 for the breaks and 30 for the density. 

An example of a graph that is not possible to plot without having the whole data set in memory is scatter plot, since we will need to know the location of all the points.

%----------------------------------------------------------------------------------------
\newpage
\section{Histogram} \label{c3.1}
Traditionally, the heights of each bin for a histogram while plotting the frequency, the height of each bin is determined by the number of values that are within the ranges of a certain bin, and plotting the density is determined by the proportion of data in each bin divided by the width of each bin.

To calculate the proportions, we first need to determine which bin each value is in. To do this we can use the {\ttfamily base::cut()} and perform {\ttfamily svydbmean()} on the cuts. However this function is not compatible with database tables, therefore a new function {\ttfamily db\_cut2} was implemented to overcome this difficulty, this will be discussed with more detail in section \ref{c3.1.3}.

\subsection{Usage} \label{c.3.1.1}
\begin{center}
    {\ttfamily svydbhist(x, design, binwidth = NULL, xlab = "x", ylab = "Density")}
\end{center}
\subsection{Arguments} 
\begin{itemize}
\item $x$ = Name indicating the variable.

\item $design$ = svydb.design object.

\item $binwidth$ = The width of each bin. Binswidths are calculated with Sturges' formula \citep{sturges} by default, $ k = [log_2 n] + 1$.}.
 
% Sturges, H. A. (1926). "The choice of a class interval". Journal of the American Statistical Association: 65–66. doi:10.1080/01621459.1926.10502161. JSTOR 2965501.

\item $xlab, ylab$ = labels for xlab and ylab.
\end{itemize}

\subsection{Examples} \label{c.3.1.2}
\begin{lstlisting}
> nh.dbsurv = svydbdesign(st = SDMVSTRA, wt = WTMEC2YR, 
    id = SDMVPSU, data = nhdb)
> svydbhist(x = DirectChol , design = nh.dbsurv,
    binwidth = 0.25)
\end{lstlisting}

\begin{figure}[h]
    \centering
    \includegraphics[scale = 0.55]{img/hist-e.jpeg}
    \caption{{\ttfamily svydbhist} example}
    \label{fig:hist-e}
\end{figure}




\subsection{Difficulty} \label{c3.1.3}
\begin{enumerate}
    \item In {\sf R}, it is simple to divide numeric columns into intervals by using {\ttfamily base::cut()}, but this function is not supported in {\sf SQL}. To overcome this problem a new function was written, {\ttfamily db\_cut2}, it is designed to be compatible with {\sf SQL} tables and is a simple version of {\ttfamily base::cut()}. 
\end{enumerate}

%----------------------------------------------------------------------------------------
\newpage
\section{Box Plot} \label{c3.2}

Boxplots are based on the quantiles of each variables or groups of variables. Traditionally, by default in {\sf R}, the boxes of the boxplot only uses the median, $25^{th}$ percent and $75^{th}$ percent quantile. The ends of the boxplots only extends out to the observations that are within the 1.5 $\times$ interquantile range, the rest of the observation outside this range are plotted as points. In {\ttfamily svydbboxplot} it is the opposite, due to efficiency.

\subsection{Usage} \label{c.3.1.1}
\begin{center}
    {\ttfamily svydbboxplot(x, groups = NULL, design, varwidth = F, outlier = F)}
\end{center}
\subsection{Arguments}
\begin{itemize}
\item $x, groups$ = If {\ttfamily groups} is defined, boxes of x will be split by {\ttfamily groups}.

\item $design$ = svydb.design object.

\item $varwidth$ = If {\ttfamily varwidth = T}, the width of the boxes will be proportional to the number of observations in that box.

\item $outlier$ = If {\ttfamily outlier = T}, Any observations above or below the $1.5IQR$ will be plotted as points.

\item $all.outlier$ = {\ttfamily TRUE} to plot all the outlier points, default {\ttfamily FALSE}.
\end{itemize}

\subsection{Examples} \label{c.3.1.2}
\begin{lstlisting}
> nh.dbsurv = svydbdesign(st = SDMVSTRA, wt = WTMEC2YR, 
    id = SDMVPSU, data = nhdb)
> svydbboxplot(x = Weight, groups = Race3, 
    design = nh.dbsurv, outlier = T, 
    all.outlier = F, varwidth = T)
\end{lstlisting}

\begin{figure}[h]
    \centering
    \includegraphics[scale = 0.55]{img/box-e.jpeg}
    \caption{{\ttfamily svydbboxplot} example}
    \label{fig:box-e}
\end{figure}

%----------------------------------------------------------------------------------------
\newpage
\section{Hexagon Binning} \label{c3.3}

As mentioned at the beginning of this chapter, since survey data sets can be large and scatter plots does not work well with large data sets because the points will hide or overlap each other. A solution would be to use hexagon binning \citep{hexbin}, group points that are close to each other into a hexagon and use colours or sizes of the hexagons to represent the sampling weights of all the points in every hexagon. The reasons of using hexagons will not be discussed here, but it is proven to be efficient and less biased visually. \\

Briefly, the hexagon binning algorithm works as follows, 
\begin{enumerate}
    \item Figure~\ref{fig:hex-e1} (Page~\pageref{fig:hex-e1}) Panel 1 - The red point represents the point to be binned. Blue and black point represents the dual lattice, each point represents the corner of each rectangle.
    
    \item Figure~\ref{fig:hex-e1} (Page~\pageref{fig:hex-e1}) Panel 2} - Figure out which rectangles are closest to the red point. The intersect of the two rectangles should contain the red point.
    
    \item Figure~\ref{fig:hex-e1} (Page~\pageref{fig:hex-e1}) Panel 3 - Figure out which hexagon the red point should be in by calculating the distance between the point and the centres of two hexagons. The centre for the blue hexagon is the top left corner of the black rectangle and the centre for the black hexagon is the bottom right corner of the blue rectangle.
    
    \item Repeat for the next point.
\end{enumerate}
%The algorithm repeats for every point.

\subsection{Usage/Arguments} \label{c.3.1.1}
To compute the hexagon bins:
\begin{center}
    {\ttfamily svydbhexbin(formula, design, xbins = 30, shape = 1)}
\end{center}

\begin{itemize}
\item $formula$ = A formula indicating x and y. i.e. y~x.

\item $design$ = svydb.design object.

\item $xbins$ = Number of bins on range of the x-axis.

\item $shape$ = plotting region, shape = height of y/width of x.
\end{itemize}

To plot the hexagon bins:
\begin{center}
    {\ttfamily svydbhexplot(d, xlab = d\$xlab, ylab = d\$ylab)}
\end{center}

\begin{itemize}
\item $d$ = returning object of {\ttfamily svydbhexbin()}.

\item $xlab, ylab$ = labels for x and y axis.
\end{itemize}

\newpage

\subsection{Examples} \label{c.3.1.2}
\begin{lstlisting}
> nh.dbsurv = svydbdesign(st = SDMVSTRA, wt = WTMEC2YR, 
    id = SDMVPSU, data = nhdb)
> hb = svydbhexbin(Height~Weight, design = nh.dbsurv)
> svydbhexplot(hb)
\end{lstlisting}


\begin{figure}[h]
    \centering
    \includegraphics[scale = 0.55]{img/hex-e.jpeg}
    \caption{{\ttfamily svydbhexplot} example}
    \label{fig:hex-e}
\end{figure}

\subsection{Difficulties}

\begin{enumerate}
    \item The original code for hexagon binning in the {\bf hexbin} {\sf R} package \citep{hexbinpackage} used for-loops to run the algorithm, but it is not possible in {\sf SQL}. Therefore, new columns and extra calculations were needed to overcome this problem. This caused a efficiency problem.

\end{enumerate}

\begin{figure}[H]
\centering
    \includegraphics[scale = 0.9]{img/hex-e1.jpg}
    \caption{Hexagon binning explanation, \citep{hexbinfig}}
    \label{fig:hex-e1}
\end{figure}




%----------------------------------------------------------------------------------------
\newpage
\section{Conditional Plots} \label{c3.4}
% ref --- cleveland and co workers in the trellis display system
Conditional plots in {\bf svydb} products sets of hexagon binning graphs conditioned by the condition given by the user, both x and y axis will remain the same for all graphs. Currently it only supports conditions applied on factors.

\subsection{Function description} \label{c.3.1.1}
The {\ttfamily svydbcoplot()} function, has three basic arguments,
\begin{center}
    {\ttfamily svydbcoplot(formula, by, design)}
\end{center}

\begin{itemize}
\item $formula$ = Formula indicating x and y. i.e. y~x.

\item $by$ = Formula indicating the conditions of each plot. i.e by1~by2.

\item $design$ = svydb.design object.
\end{itemize}

\subsection{Examples} \label{c.3.1.2}
\begin{lstlisting}
> nh.dbsurv = svydbdesign(st = SDMVSTRA, wt = WTMEC2YR, 
    id = SDMVPSU, data = nhdb)
> svydbcoplot(Age~Height, by = SmokeNow~Gender, 
    design = nh.dbsurv)
\end{lstlisting}

\begin{figure}[h]
    \centering
    \includegraphics[scale = 0.55]{img/coplot-e.jpeg}
    \caption{{\ttfamily svydbcoplot} example}
    \label{fig:coplot-e}
\end{figure}
%----------------------------------------------------------------------------------------


\newpage
\section{Why ggplot?} \label{c3.5}

There are several reasons why  graphics in {\bf svydb} is implemented with {\bf ggplot2}, one of them is that {\bf ggplot2}, {\bf dplyr} and {\bf dbplyr} is maintained by the same group of people and is a part of a project at Rstudio. Though currently, {\bf ggplot2} is only compatible with data sets in memory, but it is possible this will change in the future. Another reason could be that the user would like to have a interactive plot, this done by {\bf plotly} \citep{plotlypackage}. \\

However, the main reason is that it provides more options for the users in terms of customising the plots. 
In base graphics, it would be difficult to customise plots when the arguments provided does not include what we want, also it would be difficult for the author to consider all the arguments when creating a function.\\

A simple example is that, when a {\ttfamily gg} or a {\ttfamily ggplot} object is created,
\begin{lstlisting}
> p = ggplot(mpg, aes(class, hwy))
\end{lstlisting}
Customising on the object is just as simple as adding other functions on,
\begin{lstlisting}
> p + geom_boxplot(varwidth = T) + 
    geom_jitter(width = 0.2) + coord_flip() + 
    ggtitle("plot") + theme_bw()
\end{lstlisting}

\begin{figure}[h]
    \centering
    \includegraphics[scale = 0.55]{img/whygg1.jpeg}
    \caption{Why use {\ttfamily ggplot}}
\end{figure}

Or switching the style of the plot is as simple as changing the {\ttfamily geom},
\begin{lstlisting}
> p + geom_violin()
\end{lstlisting}

% \begin{figure}[h]
%     \centering
%     \includegraphics[scale = 0.55]{img/whygg2.jpeg}
% \end{figure}
% Chapter Template

\chapter{Results} \label{c4} % Main chapter title

In this chapter, the time it takes to compute each survey statistics with {\bf svydb} will be tested against the {\bf survey} package, both data sets in memory (local) and in database will be used for {\bf svydb}. Up to two million observations was tested with data sets in memory and up to four million observations for data sets in a data base, data set used was The National Health and Nutrition Examination Survey.\\

All computation was done with the same computer, specs as follows,
\begin{itemize}
    \item {\bf MacBook Pro Early 2015}
    \item {\bf Processor} 2.7 GHz Intel Core i5 
    \item {\bf Memory} 8 GB 1867 MHz DDR3
    \item {\bf Graphics} Intel Iris Graphics 6100 1536 MB
\end{itemize}

\hfill

Data sets from 250,000 to 2,000,000 observations were tested on functions which computes replicate survey statistics, up to 4,000,000 observations for other survey statistics and graphics. For all the times, please refer to Appendix \ref{AppendixA}.\\

Legend on graphs corresponds to,
\begin{itemize}
    \item {\bf survey} - {\bf survey} package on local data sets.
    \item {\bf svydb.local}  -  {\bf svydb} on local data sets.
    \item {\bf svydb.database} -  {\bf svydb} package on data sets stored in a database.
\end{itemize}




\section{Total}
\subsection{Numeric variable}
\begin{figure}[H]
    \centering
    \includegraphics[scale = 0.6]{Latex/img/totnum-r.jpeg}
    \caption{Population total time comparison - Numeric}
    \label{fig:totnum-r}
\end{figure}

\subsection{Categorical variable with 7 levels}

\begin{figure}[H]
    \centering
    \includegraphics[scale = 0.6]{Latex/img/totcat-r.jpeg}
    \caption{Population total time comparison - Categorical}
    \label{fig:totcat-r}
\end{figure}

%--------------------------------------------------------------------------------------
\section{Mean}
\subsection{Numeric variable}
\begin{figure}[H]
    \centering
    \includegraphics[scale = 0.6]{Latex/img/meannum-r.jpeg}
    \caption{Population mean time comparison - Numeric}
    \label{fig:meannum-r}
\end{figure}

\subsection{Categorical variable with 7 levels}
\begin{figure}[H]
   \centering
    \includegraphics[scale = 0.6]{Latex/img/meancat-r.jpeg}
    \caption{Population mean time comparison - Categorical}
    \label{fig:meancat-r}
\end{figure}

 

%--------------------------------------------------------------------------------------
\section{Regression}
\begin{figure}[H]
   \centering
    \includegraphics[scale = 0.6]{Latex/img/lm-r.jpeg}
    \caption{Regression time comparison - 5 variables}
    \label{fig:lm-r}
\end{figure}
%--------------------------------------------------------------------------------------
\section{Quantiles}
\begin{figure}[H]
   \centering
    \includegraphics[scale = 0.6]{Latex/img/quantile-r.jpeg}
    \caption{Median time comparison - Categorical}
    \label{fig:quantile-r}
\end{figure}
%--------------------------------------------------------------------------------------
\section{Survey Tables}
\begin{figure}[H]
   \centering
    \includegraphics[scale = 0.6]{Latex/img/table-r.jpeg}
    \caption{Tables time comparison - Categorical}
    \label{fig:table-r}
\end{figure}




%--------------------------------------------------------------------------------------
\section{Total - Replicate Weights}
\begin{figure}[H]
  \centering
    \includegraphics[scale = 0.6]{Latex/img/reptot-r.jpeg}
    \caption{Replicate total time comparison - Numeric}
    \label{fig:reptot-r}
\end{figure}
%--------------------------------------------------------------------------------------
\section{Mean - Replicate Weights}
\begin{figure}[H]
  \centering
    \includegraphics[scale = 0.6]{Latex/img/repmean-r.jpeg}
    \caption{Replicate mean time comparison - Numeric}
    \label{fig:repmean-r}
\end{figure}



%--------------------------------------------------------------------------------------
\section{Histogram}
\begin{figure}[H]
   \centering
    \includegraphics[scale = 0.6]{Latex/img/hist-r.jpeg}
    \caption{Histogram time comparison - Categorical}
    \label{fig:hist-r}
\end{figure}
%--------------------------------------------------------------------------------------
\section{Boxplot}
\begin{figure}[H]
   \centering
    \includegraphics[scale = 0.6]{Latex/img/box-r.jpeg}
    \caption{Boxplot time comparison}
    \label{fig:box-r}
\end{figure}
%--------------------------------------------------------------------------------------
\section{Hexagon Binning}
\begin{figure}[H]
   \centering
    \includegraphics[scale = 0.6]{Latex/img/hex-r.jpeg}
    \caption{Hexagon Binning time comparison}
    \label{fig:hex-r}
\end{figure}
%--------------------------------------------------------------------------------------

\newpage

\section{Time Comparison}

In general, using {\bf svydb} with local data sets is always faster than both {\bf svydb} with data sets in a database or with the {\bf survey} package, except for computing the survey tables (\autoref{fig:table-r}) and hexagon binning (\autoref{fig:hex-r}), where the times were similar for survey tables and hexagon binning was always about 0.5 seconds slower.\\

In terms of data sets in a database, If using {\bf svydb} with local data sets is faster than the {\bf survey} package then the time it takes to compute survey statistics in a database with {\bf svydb} is predicted to be faster than the {\bf survey} package if the data set is large enough. When the data set gets larger, the increase in time for the {\bf survey} package is almost always larger than {\bf svydb}. In some cases with data sets up to two million observations, {\bf svydb} is already faster. \\

Most of the time we can see that {\bf svydb} with data sets in memory is the fastest, followed by the {\bf survey} package, and {\bf svydb} with data sets in a database is always the slowest. The reason is that it takes time for the computer to communicate with the database, as in telling it what to do and collecting the results. However, if the data sets are too large and cannot be fitted into memory, {\bf svydb} would be useful.\\

Some functions are just a replicate of another function but with more iterations and more features. For example, within {\bf svydbboxplot()} (\autoref{fig:box-r}), {\bf svydbquantile()} (\autoref{fig:quantile-r}) is called to obtain the quantiles, {\bf svydbhist()} (\autoref{fig:hist-r}) calls {\bf svydbmean()} (\autoref{fig:meannum-r}, \autoref{fig:meancat-r}) to obtain the proportions of each cut, and {\bf svydbreptotal()} (\autoref{fig:reptot-r}) is computed similarly to {\bf svydbtotal()} (\autoref{fig:totnum-r}, \autoref{fig:totcat-r}) but with more iterations with different sets of weights. Therefore, their computation time would be similar across different sizes of data set but with some sort of a scale factor.\\

Regression (\autoref{fig:lm-r}) is the most interesting, though the slope of the line for {\bf svydb.database} seems to be decreasing at 1.75 million observations, it is still around 1 to 1.5 seconds slower than the {\bf survey} package at the two million observations mark. Since matrix operations are not supported in a database, to get the result of a matrix multiplication is to calculate it row by row which would be slow. Therefore {\bf svydb.database} would only be expected to be faster than the {\bf survey} package if the data set is large enough, or if the model contains enough explanatory variables which results in an unfeasible matrix operation.\\

Though it seems as if {\bf svydb} is faster in most cases, but the structure of the {\bf survey} package is more complex, and what it can do is outside the capability of {\bf svydb}, for example, handling missing values, post-stratification and log models. At this stage it is unclear how much faster the {\bf survey} package would be if was only computing statistics that are implemented in {\bf svydb}.











 
% Chapter Template

\chapter{Usability} \label{c5} % Main chapter title

\section{The package}
With the help of the {\bf devtools} package \citep{devtoolspackage} for package building, the {\bf roxygen2}  package \citep{roxygen2package} for package documentation amd the {\bf formatR} \citep{formatrpackage} package for formatting. The full {\bf svydb} package has been uploaded and can be found on \href{https://github.com/chrk623/svydb}{Github}, where it also includes basic help pages for each functions that computes survey statistics.\\

The dependant packages of {\bf svydb} are,
{\bf
    \begin{itemize}
        \item R6 \citep{R6package}
        \item dplyr \citep{dplyrpackage}
        \item rlang \citep{rlangpackage}
        \item survey \citep{surveypackage}
        \item ggplot2 \citep{ggplotpackage}
    \end{itemize}
}



\section{Supporting functions}
Along with the main functions which computes survey statistics, there are also a number of other functions that were written to help with the implementation of the main functions, testing those functions and to overcome difficulties where there are inconsistency in terms of how to query between different types of databases. Details of these functions can be found in Appendix \ref{appena:otherfunc}.

\section{Testing}


Examples within the {\bf survey} package help page was used to test {\bf svydb}, all results were identical, except for when the data sets are large enough that it causes rounding error. When examples were not available for certain functions, alternatives were used.
 
%% Chapter Template
%
\chapter{Discussion} \label{c6}

At the beginning of the report, we have discussed the limitations of a few current approaches of teaching data joining and reshaping. The traditional method of teaching with static images is limited because they are not descriptive enough. First, they usually use dummy data sets with no meaning, or data sets with vocabulary that the learners may be unfamiliar with. Second, they assume that learners can imagine data sets or the process of the transformation in their head.

Another approach found was to teach this with animations. With this approach, the user does not have to imagine the data transformation process. However, they still use data sets which are meaningless. Although, this does solve some of the limitations that the static image approach brings, we are interested to extend this.

Therefore, we attempted develop a tool which generate animations to address all the limitations mentioned above. Additionally, this tool can also be used as a resource for the data joining and reshaping module in \textbf{iNZight} to help users understand these operations in visual way.


\section{Future work}
Though some of the animations in the software already handles complicated situations while joining or reshaping data sets, we can still expand them to handle more complicated tasks. 

Additionally, we should allow students or learners to view these animations and make adjustments to the software based on their feedback.

Lastly, a \textsf{SQL} module is already under development. Although it is not ready at the moment but it is a good extension to our current joining module. 



\section{Conclusion}
We have seen multiple approaches of teaching data joining and reshaping, they turned out to be not as efficient, mostly because of the of lack explanation in the transformation process. Therefore to overcome this problem, we used technique like highlighting, line animation, fading and instructional messages. On top of that we tried to be very careful with the ordering of these operations. 

This piece of software allows users to visualise the process of data joining and reshaping. Therefore, they should be more easier to understand than most of the common approach of teaching data joining and reshaping.


\newpage
\section{Usage}
The \textbf{dataAnim} is hosted on \textsf{Github} at \href{https://github.com/chrk623/dataAnim}{https://github.com/chrk623/dataAnim}. To install \textbf{dataAnim} in \textsf{R}:

\begin{lstlisting}
 devtools::install_github("chrk623\dataAnim")
\end{lstlisting}

It is suggested to view the animations in \textbf{Chrome} and have the latest version of \textbf{Rstudio}. It is also suggested to use the most up to date version of all the dependency packages.

Lastly, the shiny interactive dashboard can be found at 

\href{https://chrk623.shinyapps.io/dataAnim_shiny/}{https://chrk623.shinyapps.io/dataAnim\_shiny/}.


 
%----------------------------------------------------------------------------------------
%	THESIS CONTENT - APPENDICES
%----------------------------------------------------------------------------------------

\appendix % Cue to tell LaTeX that the following "chapters" are Appendices

% Include the appendices of the thesis as separate files from the Appendices folder
% Uncomment the lines as you write the Appendices

% Appendix Template

\chapter{Result Tables} \label{AppendixA}

{\bf All units are in seconds.}

\section{Total}
\subsection{Numeric variable}
\begin{table}[ht]
\centering
\begin{tabular}{rrrrr}
  \hline
 & Observations & Survey & svydb.local & svydb.database \\ 
  \hline
1 & 250000 & 0.43 & 0.06 & 0.40 \\ 
  2 & 500000 & 0.88 & 0.08 & 0.41 \\ 
  3 & 750000 & 1.38 & 0.11 & 0.61 \\ 
  4 & 1000000 & 1.76 & 0.15 & 0.72 \\ 
  5 & 1250000 & 2.35 & 0.17 & 0.81 \\ 
  6 & 1500000 & 2.80 & 0.21 & 0.94 \\ 
  7 & 1750000 & 3.26 & 0.28 & 0.97 \\ 
  8 & 2000000 & 3.95 & 0.26 & 1.15 \\ 
  9 & 2500000 & - & - & 1.17 \\ 
  10 & 3000000 & - & - & 1.56 \\ 
  11 & 3500000 & - & - & 1.61 \\ 
  12 & 4000000 & - & - & 1.92 \\ 
   \hline
\end{tabular}
\end{table}

\subsection{Categorical variable with 7 levels}
\begin{table}[ht]
\centering
\begin{tabular}{rrrrr}
  \hline
 & Observations & Survey & svydb.local & svydb.database \\ 
  \hline
1 & 250000 & 0.73 & 0.22 & 1.20 \\ 
  2 & 500000 & 1.25 & 0.28 & 1.50 \\ 
  3 & 750000 & 1.87 & 0.34 & 1.76 \\ 
  4 & 1000000 & 2.58 & 0.46 & 2.03 \\ 
  5 & 1250000 & 3.10 & 0.51 & 2.28 \\ 
  6 & 1500000 & 3.94 & 0.61 & 2.65 \\ 
  7 & 1750000 & 4.42 & 0.71 & 2.92 \\ 
  8 & 2000000 & 5.27 & 0.79 & 3.20 \\ 
  9 & 2500000 & - & - & 3.63 \\ 
  10 & 3000000 & - & - & 4.25 \\ 
  11 & 3500000 & - & - & 4.67 \\ 
  12 & 4000000 & - & - & 5.21 \\ 
   \hline
\end{tabular}
\end{table}

%--------------------------------------------------------------------------------------
\section{Mean}
\subsection{Numeric variable}
\begin{table}[ht]
\centering
\begin{tabular}{rrrrr}
  \hline
 & Observations & Survey & svydb.local & svydb.database \\ 
  \hline
1 & 250000 & 0.43 & 0.09 & 0.63 \\ 
  2 & 500000 & 0.90 & 0.11 & 0.79 \\ 
  3 & 750000 & 1.35 & 0.14 & 0.96 \\ 
  4 & 1000000 & 1.86 & 0.21 & 1.24 \\ 
  5 & 1250000 & 2.13 & 0.23 & 1.31 \\ 
  6 & 1500000 & 2.77 & 0.27 & 1.48 \\ 
  7 & 1750000 & 3.26 & 0.30 & 1.62 \\ 
  8 & 2000000 & 3.60 & 0.34 & 1.85 \\ 
  9 & 2500000 & - & - & 2.09 \\ 
  10 & 3000000 & - & - & 2.39 \\ 
  11 & 3500000 & - & - & 2.66 \\ 
  12 & 4000000 & - & - & 3.09 \\ 
   \hline
\end{tabular}
\end{table}

\subsection{Categorical variable with 7 levels}
\begin{table}[ht]
\centering
\begin{tabular}{rrrrr}
  \hline
 & Observations & Survey & svydb.local & svydb.database \\ 
  \hline
1 & 250000 & 0.72 & 0.21 & 1.65 \\ 
  2 & 500000 & 1.25 & 0.35 & 2.49 \\ 
  3 & 750000 & 1.90 & 0.47 & 2.73 \\ 
  4 & 1000000 & 2.49 & 0.70 & 3.26 \\ 
  5 & 1250000 & 3.26 & 0.76 & 3.83 \\ 
  6 & 1500000 & 4.02 & 0.89 & 4.50 \\ 
  7 & 1750000 & 4.85 & 1.02 & 5.04 \\ 
  8 & 2000000 & 5.55 & 0.94 & 5.63 \\ 
  9 & 2500000 & - & - & 6.68 \\ 
  10 & 3000000 & - & - & 7.53 \\ 
  11 & 3500000 & - & - & 8.70 \\ 
  12 & 4000000 & - & - & 9.72 \\ 
   \hline
\end{tabular}
\end{table}

\newpage 

%--------------------------------------------------------------------------------------
\section{Regression}
\begin{table}[ht]
\centering
\begin{tabular}{rrrrr}
  \hline
 & Observations & Survey & svydb.local & svydb.database \\ 
  \hline
1 & 250000 & 1.18 & 0.68 & 2.95 \\ 
  2 & 500000 & 2.00 & 1.03 & 4.24 \\ 
  3 & 750000 & 3.16 & 1.30 & 5.31 \\ 
  4 & 1000000 & 4.50 & 1.31 & 6.56 \\ 
  5 & 1250000 & 5.65 & 1.65 & 7.75 \\ 
  6 & 1500000 & 6.77 & 1.97 & 8.70 \\ 
  7 & 1750000 & 8.13 & 2.24 & 10.09 \\ 
  8 & 2000000 & 9.75 & 2.68 & 10.72 \\ 
  9 & 2500000 & - & - & 12.31 \\ 
  10 & 3000000 & - & - & 14.31 \\ 
  11 & 3500000 & - & - & 16.32 \\ 
  12 & 4000000 & - & - & 18.22 \\ 
   \hline
\end{tabular}
\end{table}

%--------------------------------------------------------------------------------------
\section{Quantiles}
\begin{table}[ht]
\centering
\begin{tabular}{rrrrr}
  \hline
 & Observations & Survey & svydb.local & svydb.database \\ 
  \hline
1 & 250000 & 0.58 & 0.08 & 0.36 \\ 
  2 & 500000 & 1.16 & 0.18 & 0.51 \\ 
  3 & 750000 & 1.55 & 0.26 & 0.63 \\ 
  4 & 1000000 & 2.18 & 0.29 & 0.83 \\ 
  5 & 1250000 & 2.80 & 0.28 & 0.93 \\ 
  6 & 1500000 & 3.45 & 0.31 & 1.04 \\ 
  7 & 1750000 & 4.24 & 0.27 & 1.17 \\ 
  8 & 2000000 & 4.82 & 0.32 & 1.40 \\ 
  9 & 2500000 & - & - & 1.49 \\ 
  10 & 3000000 & - & - & 1.73 \\ 
  11 & 3500000 & - & - & 1.98 \\ 
  12 & 4000000 & - & - & 2.25 \\ 
  \hline
\end{tabular}
\end{table}

\newpage

%--------------------------------------------------------------------------------------
\section{Survey Tables}
\begin{table}[ht]
\centering
\begin{tabular}{rrrrr}
  \hline
 & Observations & Survey & svydb.local & svydb.database \\ 
  \hline
1 & 250000 & 0.18 & 0.26 & 0.67 \\ 
  2 & 500000 & 0.39 & 0.49 & 1.02 \\ 
  3 & 750000 & 0.56 & 0.69 & 1.35 \\ 
  4 & 1000000 & 0.78 & 0.76 & 1.69 \\ 
  5 & 1250000 & 0.97 & 1.00 & 2.05 \\ 
  6 & 1500000 & 1.16 & 1.11 & 2.44 \\ 
  7 & 1750000 & 1.38 & 1.37 & 2.56 \\ 
  8 & 2000000 & 1.56 & 1.47 & 2.97 \\ 
  9 & 2500000 & - & - & 3.45 \\ 
  10 & 3000000 & - & - & 4.05 \\ 
  11 & 3500000 & - & - & 4.63 \\ 
  12 & 4000000 & - & - & 5.20 \\ 
  \hline
\end{tabular}
\end{table}




%--------------------------------------------------------------------------------------
\section{Total - Replicate Weights}
\begin{table}[ht]
\centering
\begin{tabular}{rrrrr}
  \hline
 & Observations & Survey & svydb.local & svydb.database \\ 
  \hline
1 & 250000 & 3.54 & 0.46 & 2.12 \\ 
  2 & 500000 & 9.03 & 0.74 & 3.49 \\ 
  3 & 750000 & 12.31 & 1.13 & 4.81 \\ 
  4 & 1000000 & 21.57 & 1.45 & 5.86 \\ 
  5 & 1250000 & - & - & 7.10 \\ 
  6 & 1500000 & - & - & 9.34 \\ 
  7 & 1750000 & - & - & 10.62 \\ 
  8 & 2000000 & - & - & 14.75 \\ 
  \hline
\end{tabular}
\end{table}


%--------------------------------------------------------------------------------------
\section{Mean - Replicate Weights}
\begin{table}[ht]
\centering
\begin{tabular}{rrrrr}
  \hline
 & Observations & Survey & svydb.local & svydb.database \\ 
  \hline
1 & 250000 & 3.57 & 0.66 & 3.66 \\ 
  2 & 500000 & 8.69 & 0.98 & 5.68 \\ 
  3 & 750000 & 14.25 & 1.61 & 7.23 \\ 
  4 & 1000000 & 20.27 & 1.89 & 8.72 \\ 
  5 & 1250000 & - & - & 10.36 \\ 
  6 & 1500000 & - & - & 12.24 \\ 
  7 & 1750000 & - & - & 13.97 \\ 
  8 & 2000000 & - & - & 16.02 \\ 
  \hline
\end{tabular}
\end{table}

\newpage

%--------------------------------------------------------------------------------------
\section{Histogram}
\begin{table}[ht]
\centering
\begin{tabular}{rrrrr}
  \hline
 & Observations & Survey & svydb.local & svydb.database \\ 
  \hline
1 & 250000 & 0.97 & 0.58 & 1.57 \\ 
  2 & 500000 & 2.07 & 0.75 & 2.56 \\ 
  3 & 750000 & 3.14 & 0.90 & 2.84 \\ 
  4 & 1000000 & 4.03 & 1.05 & 3.49 \\ 
  5 & 1250000 & 5.15 & 1.15 & 4.12 \\ 
  6 & 1500000 & 6.12 & 1.42 & 4.79 \\ 
  7 & 1750000 & 7.50 & 1.48 & 5.24 \\ 
  8 & 2000000 & 8.57 & 1.66 & 6.05 \\ 
  9 & 2500000 & - & - & 7.03 \\ 
  10 & 3000000 & - & - & 7.94 \\ 
  11 & 3500000 & - & - & 9.15 \\ 
  12 & 4000000 & - & - & 10.34 \\ 
  \hline
\end{tabular}
\end{table}

%--------------------------------------------------------------------------------------
\section{Boxplot}
\begin{table}[ht]
\centering
\begin{tabular}{rrrrr}
  \hline
 & Observations & Survey & svydb.local & svydb.database \\ 
  \hline
1 & 250000 & 2.84 & 1.21 & 4.37 \\ 
  2 & 500000 & 4.96 & 1.39 & 5.20 \\ 
  3 & 750000 & 7.44 & 1.54 & 6.16 \\ 
  4 & 1000000 & 10.36 & 1.69 & 7.01 \\ 
  5 & 1250000 & 12.83 & 1.90 & 7.88 \\ 
  6 & 1500000 & 14.96 & 2.01 & 8.92 \\ 
  7 & 1750000 & 18.22 & 2.14 & 9.61 \\ 
  8 & 2000000 & 22.12 & 2.28 & 10.31 \\ 
  9 & 2500000 & - & - & 11.38 \\ 
  10 & 3000000 & - & - & 13.01 \\ 
  11 & 3500000 & - & - & 14.56 \\ 
  12 & 4000000 & - & - & 16.20 \\ 
  \hline
\end{tabular}
\end{table}

\newpage
%--------------------------------------------------------------------------------------
\section{Hexagon Binning}
\begin{table}[ht]
\centering
\begin{tabular}{rrrrr}
  \hline
 & Observations & Survey & svydb.local & svydb.database \\ 
  \hline
1 & 250000 & 0.41 & 0.92 & 2.05 \\ 
  2 & 500000 & 0.56 & 1.23 & 3.23 \\ 
  3 & 750000 & 0.99 & 1.53 & 4.27 \\ 
  4 & 1000000 & 1.00 & 1.97 & 5.53 \\ 
  5 & 1250000 & 1.50 & 2.17 & 6.40 \\ 
  6 & 1500000 & 1.54 & 2.47 & 7.33 \\ 
  7 & 1750000 & 1.97 & 2.61 & 8.24 \\ 
  8 & 2000000 & 2.32 & 2.92 & 9.29 \\ 
  9 & 2500000 & - & - & 10.65 \\ 
  10 & 3000000 & - & - & 12.59 \\ 
  11 & 3500000 & - & - & 14.41 \\ 
  12 & 4000000 & - & - & 16.40 \\ 
  \hline
\end{tabular}
\end{table}
%--------------------------------------------------------------------------------------

% Appendix A

\chapter{Codes} \label{AppendixB}

\section{svydbdesign}
\begin{lstlisting}
makesvydbdesign <- R6Class("svydb.design", 
                            public = list(dataOg = NULL, 
                            data = NULL, dataSub = NULL, 
                            vars = NULL, st = NULL, 
                            id = NULL, wt = NULL, 
                            call = NULL, names = list(), 
                            levels = list(), 
    initialize = function(vars = NA, 
        st = NA, id = NA, wt = NA, data) {
        if (quo_is_null(wt)) {
            stop("Please provide sampling weights")
        } else {
            self$wt = as.character(wt)[2]
        }
        
        if (quo_is_null(st)) {
            data = data %>% mutate(st = 1)
            self$st = "st"
        } else {
            self$st = as.character(st)[2]
        }
        
        if (quo_is_null(id)) {
            data = data %>% mutate(id = row_number())
            self$id = "id"
        } else {
            self$id = as.character(id)[2]
        }
        self$data = data %>% select(everything())
        self$dataOg <<- self$data
    }, 
    setx = function(x) {
        tc = tryCatch(class(x), error = function(e) e)
        
        if ("formula" %in% tc) {
            x = all.vars(x)
            self$data <<- self$data %>% 
                select(!!x, self$st, self$id, self$wt) %>% 
                filter_all(any_vars(!is.na(.)))
            self$vars <<- x
        } else {
            x = enquo(x)
            self$data <<- self$data %>% 
                select(!!x, self$st, self$id, self$wt) %>% 
                filter(!is.na(!!x))
            self$vars <<- as.character(x)[2]
        }
        self$names[["logged"]] = 
            c(self$st, self$id, self$wt, "m_h")
    }, 
    addx = function(x) {
        l = enquo(x)
        r = syms(colnames(self$data))
        self$data = self$dataOg %>% select(!!l, !!!r)
    }, 
    getwt = function() {
        self$data %>% select(self$wt) %>% 
            summarise_all(sum) %>% pull()
    }, 
    getmh = function() {
        self$data %>% group_by(!!sym(self$st)) %>% 
            summarise(m_h = n_distinct(!!sym(self$id)))
    }, 
    subset = function(..., logical = T) {
        d = self$clone()
        
        if (logical == T) {
            d$data = d$data %>% filter(...)
        } else {
            d$data = d$data %>% filter(!!parse_expr(...))
        }
        return(d)
    }, 
    subset_rows = function(from, to) {
        self$dataSub = self$data %>% 
            db_selectRows(., from = from, to = to)
    }, 
    storename = function(name, obj, force = FALSE) {
        if (force == TRUE) {
            self$names$logged = 
                self$names$logged[-which(
                        self$names$logged %in% obj)]
        }
        if (!all(obj %in% self$names$logged)) {
            new = setdiff(obj, self$names$logged)
            self$names[[name]] = c(new)
            self$names$logged = c(self$names$logged, new)
        }
        
    }, 
    removename = function(name, obj) {
        self$names$logged = 
            self$names$logged[-which(
                self$names$logged %in% obj)]
        self$names[[name]] = 
            (self$names[[name]])[-which(
                self$names[[name]] %in% obj)]
    }, 
    storelevel = function(x, name) {
        ll = list(x)
        names(ll) = name
        self$levels = c(self$levels, ll)
    }, 
    storecall = function(x) {
        self$call = x
    }, 
    print = function() {
        nid = self$getmh() %>% pull(m_h) %>% sum()
        rows = self$data %>% db_nrow()
        txt = sprintf("svydb.design, 
            %s observation(s), %s Clusters\n", 
            rows, nid)
        cat(txt)
    }))

svydbdesign = function(st = NULL, id = NULL, 
                            wt = NULL, data){
  st = enquo(st)
  id = enquo(id)
  wt = enquo(wt)

  d = makesvydbdesign$new(st = st, id = id, wt = wt,
                          data = data)
  d$storecall(match.call())
  d
}
\end{lstlisting}}

%----------------------------------------------------------------
\section{svydbtotal}

\begin{lstlisting}
svydbtotal = function(x, num, design, 
                     return.total = F, ...) {

    if (!("svydb.design" %in% class(design))) {
        stop("Please provide a svydb.design")
    }

    if (missing(num)) {
        stop("Is x a numeric or categorical variable?
            , num = T OR num = F?")
    }

    dsn = design$clone()
    dsn$setx(!!enquo(x))
    d = dsn$data
    dsn$storename("x", colnames(d))

    if (num == F) {
        d = dummy_mut(d, !!sym(dsn$names$x), withBase = T)
    }
    dsn$storename("x", colnames(d))
    d = d %>% mutate_at(vars(dsn$names$x), 
                        funs((. * !!sym(dsn$wt)))) %>% 
            compute(temporary = T)

    totTbl = d %>% select(dsn$names$x) %>% 
                summarise_all(sum) %>% collect() %>% t()

    if (return.total == TRUE) {
        colnames(totTbl) = "Total"
        return(totTbl)
    }

    varTbl = d %>% select(dsn$st, dsn$id, dsn$names$x) %>% 
                    group_by(!!!syms(c(dsn$st, dsn$id))) %>% 
                    summarise_at(vars(dsn$names$x),
                                funs(sum(.))) %>% 
                    compute(temporary = T)
    varTbl = inner_join(varTbl, dsn$getmh(), by = dsn$st)

    barTbl = varTbl %>% select(-one_of(dsn$id)) %>% 
                        group_by(!!sym(dsn$st)) %>% 
                        summarise_at(vars(dsn$names$x), 
                                    funs(bar = sum(./m_h)))
    dsn$storename("bar", colnames(barTbl))

    varTbl = inner_join(varTbl, barTbl, by = dsn$st)
    `zhi-zbar` = paste(dsn$names$x, "-", 
                       dsn$names$bar, collapse = " ; ")
    varTbl = varTbl %>% 
        mutate(!!!parse_exprs(`zhi-zbar`)) %>% 
        ungroup() %>% compute(temporary = T)
    dsn$storename("diff", colnames(varTbl))

    varTbl = sapply(dsn$names$diff, svydbVar, 
                    st = dsn$st, m_h = "m_h", data = varTbl)

    class(totTbl) = "svydbstat"
    attr(totTbl, "var") = varTbl
    attr(totTbl, "statistic") <- "Total"
    attr(totTbl, "name") = dsn$names$x

    return(totTbl)
}
\end{lstlisting}

%----------------------------------------------------------------
\section{svydbmean}

\begin{lstlisting}
svydbmean = function(x, num, design, 
                    return.mean = F, ...) {

    if (!("svydb.design" %in% class(design))) {
        stop("Please provide a svydb.design")
    }

    if (missing(num)) {
        stop("Is x a numeric or categorical variable?, 
             = T OR num = F?")
    }

    dsn = design$clone()
    dsn$setx(!!enquo(x))
    d = dsn$data
    dsn$storename("x", colnames(d))

    if (num == F) {
        d = dummy_mut(d, !!sym(dsn$names$x), withBase = T)
    }

    dsn$storename("x", colnames(d))
    N = dsn$getwt()
    meanTbl = d %>% 
              transmute_at(vars(dsn$names$x), 
                           funs((. * !!sym(dsn$wt))
                                        /!!quo(N))) %>% 
              summarise_all(sum) %>%
              compute(temporary = T) %>% collect()

    if (return.mean == TRUE) {
        colnames(meanTbl) = dsn$names$x
        return(meanTbl)
    }

    dhi_exprs = paste(dsn$names$x, " - ", "meanTbl$", 
                      dsn$names$x, sep = "", 
                      collapse = " ; ")
    varTbl = d %>% mutate(dhi = !!!parse_exprs(dhi_exprs))
    dsn$storename("dhi", colnames(varTbl))

    varTbl = varTbl %>% 
                mutate_at(vars(dsn$names$dhi), 
                funs((. * !!sym(dsn$wt))/!!quo(N))) %>% 
                select(dsn$st, dsn$id,
                dsn$names$dhi)

    barTbl = varTbl %>% select(dsn$st, dsn$names$dhi) %>% 
             group_by(!!sym(dsn$st)) %>% summarise_all(sum)
    barTbl = inner_join(barTbl, dsn$getmh(), 
        by = dsn$st) %>% 
        mutate_at(vars(dsn$names$dhi),
        funs(bar = ./m_h)) %>%
        select(-one_of(dsn$names$dhi))
    dsn$storename("bar", colnames(barTbl))

    varTbl = varTbl %>% 
             group_by(!!!syms(c(dsn$st, dsn$id))) %>% 
             summarise_all(sum) %>% compute(temporary = T)
    varTbl = inner_join(varTbl, barTbl, by = dsn$st)

    `dhi-dbar` = paste("`", dsn$names$dhi, "`", "-", 
                       "`", dsn$names$bar, "`", 
                       collapse = " ; ", sep = "")
    varTbl = varTbl %>% 
             mutate(!!!parse_exprs(`dhi-dbar`)) %>% 
             ungroup() %>% compute(temporary = T)
    dsn$storename("diff", colnames(varTbl))

    varTbl = sapply(dsn$names$diff, svydbVar, 
                    st = dsn$st, m_h = "m_h", data = varTbl)

    meanTbl = t(meanTbl)
    class(meanTbl) = "svydbstat"
    attr(meanTbl, "var") = varTbl
    attr(meanTbl, "statistic") <- "Mean"
    attr(meanTbl, "name") = dsn$names$x

    return(meanTbl)
}

\end{lstlisting}

%----------------------------------------------------------------
\section{svydblm}

\begin{lstlisting}
svydblm = function(formula, design) {
    if (!("svydb.design" %in% class(design))) {
        stop("Please provide a svydb.design")
    }
    
    if (missing(formula)) {
        stop("Please provide a formula")
    }
    
    fit = list()
    fit$call = match.call()
    dsn = design$clone()
    dsn$setx(formula)
    d = dsn$data
    dsn$storename("y", all.vars(formula)[1])
    dsn$storename("variables", all.vars(formula)[-1])
    
    d = d %>% mutate(`:=`(!!sym(dsn$wt), 
        case_when(is.na(!!sym(dsn$names$y)) ~ 
        0, TRUE ~ !!sym(dsn$wt))))
    d = d %>% mutate(`:=`(!!sym(dsn$names$y), 
        case_when(is.na(!!sym(dsn$names$y)) ~ 
            0, TRUE ~ !!sym(dsn$names$y))))
    d = d %>% filter_all(all_vars(!is.na(.)))
    d = d %>% mutate(intercept = 1)
    dsn$storename("intercept", colnames(d))
    xy = d
    
    facVar = attr(terms(formula, specials = "factor"), 
        "specials")$factor
    if (!is.null(facVar)) {
        facVar = facVar - 1
        dsn$storename("factor", dsn$names$variables[facVar], 
            force = T)
        for (i in 1:length(facVar)) {
            dd = dummy_mut(xy, by = 
                !!sym(dsn$names$factor[i]), 
                withBase = F, return.level = T)
            xy = dd$dum
            dsn$storelevel(x = dd$levels, 
                name = dsn$names$factor[i])
            dsn$names$variables = c(dsn$names$variables, 
                dsn$names$factor[i])
            dsn$names$variables = 
                dsn$names$variables[-(grep(
                dsn$names$factor[i], 
                dsn$names$variables)[1])]
        }
    }
    
    fit$terms = terms(paste("~", paste(dsn$names$variables, 
        collapse = " + ")) %>% as.formula)
    dsn$storename("dummy", colnames(xy))
    xy = compute(xy)
    dsn$storename("variables", c(dsn$names$variables, 
        dsn$names$dummy), force = T)
    
    if (!is.null(facVar)) {
        dsn$removename("variables", dsn$names$factor)
    }
    db_xtwx_i = function(x, col, wt, data) {
        data = data %>% summarise_at(vars(x), 
            funs(xtx = sum(. * (!!sym(col)) * 
                (!!sym(wt)))))
        return(data)
    }
    xtx = lapply(c(dsn$names$intercept, 
        dsn$names$variables), db_xtwx_i, 
        x = c(dsn$names$intercept, 
        dsn$names$variables), wt = dsn$wt, 
        data = xy)
    xtx = Reduce(full_join, xtx) %>% collect() %>% 
        as.matrix()
    colnames(xtx) = 
        c(dsn$names$intercept, dsn$names$variables)
    
    xty = xy %>% transmute_at(vars(dsn$names$intercept, 
        dsn$names$variables), funs(. * (!!sym(dsn$wt)) * 
        (!!sym(dsn$names$y)))) %>% summarise_all(sum) %>% 
        collect() %>% t()
    
    xy = xy %>% filter(!(!!sym(dsn$wt)) == 0) %>% 
        compute()
    beta = solve(xtx) %*% xty
    fit$coefname = 
        c(dsn$names$intercept, dsn$names$variables)
    fit$coefficients = beta %>% t()
    
    
    dsn$storename("xb", paste(dsn$names$variables, 
        "_xb", sep = ""))
    e = paste(dsn$names$y, " - ", beta["intercept", 
        ], " - ", paste(rownames(beta)[-1], " * ", 
        beta[-1, ], sep = "", collapse = " - "), 
        sep = "")
    
    xy = xy %>% mutate(residuals = !!parse_expr(e)) %>% 
        select(-one_of(dsn$names$y))
    dsn$storename("residuals", colnames(xy))
    res = xy %>% select(residuals)
    fit$residuals = res
    
    varTbl = xy %>% mutate_at(vars(dsn$names$intercept, 
        dsn$names$variables), 
        funs(. * (!!sym(dsn$names$residuals)) * 
        (!!sym(dsn$wt)))) %>% group_by(!!sym(dsn$st), 
        !!sym(dsn$id)) %>% summarise_all(sum) %>% 
        compute()
    dsn$storename("zhi", c(dsn$names$intercept, 
        dsn$names$variables), force = T)
    barTbl = varTbl %>% select(dsn$st, dsn$names$zhi) %>% 
        group_by(!!sym(dsn$st)) %>% summarise_all(sum)
    
    mh = dsn$getmh()
    barTbl = inner_join(barTbl, mh, by = dsn$st) %>% 
        mutate_at(vars(dsn$names$intercept, 
        dsn$names$variables), funs(bar = ./m_h)) %>% 
        compute()
    dsn$storename("bar", colnames(barTbl))
    
    varTbl = inner_join(varTbl %>% select(dsn$st, 
        dsn$names$zhi), barTbl %>% select(dsn$st, 
        dsn$names$bar, m_h), by = dsn$st)
    `zhi-zbar` = paste("`", dsn$names$zhi, "`", 
        "-", "`", dsn$names$bar, "`", collapse = " ; ", 
        sep = "")
    varTbl = varTbl %>% 
        mutate(!!!parse_exprs(`zhi-zbar`)) %>% 
        ungroup() %>% compute()
    dsn$storename("diff", colnames(varTbl))
    
    covDiag = sapply(dsn$names$diff, svydbVar, 
        st = dsn$st, m_h = "m_h", data = varTbl)
    e = outer(dsn$names$diff, dsn$names$diff, 
        paste, sep = ",")
    e = e[lower.tri(e)] %>% strsplit(., ",")
    covLt = sapply(e, svydbVar2, st = dsn$st, 
        m_h = "m_h", data = varTbl)
    cov.mat = diag(covDiag)
    cov.mat[lower.tri(cov.mat)] = covLt
    cov.mat[upper.tri(cov.mat)] = 
        t(cov.mat)[upper.tri(cov.mat)]
    
    fit$cov.unscaled = solve(xtx) %*% cov.mat %*% 
        solve(xtx)
    colnames(fit$cov.unscaled) = rownames(fit$cov.unscaled)
    
    fit$formula = formula
    fit$df.residual = sum(mh %>% select(m_h) %>% 
        pull) - db_nrow(mh) - nrow(beta) + 1
    fit$call = match.call()
    fit$design = dsn
    
    class(fit) = c("svydblm", "lm")
    return(fit)
}
\end{lstlisting}

%----------------------------------------------------------------
\section{svydbquantile}

\begin{lstlisting}
svydbquantile = function(x, quantiles = 0.5, design) {
    oldoptions = options("survey.lonely.psu")
    options(survey.lonely.psu = "adjust")
    
    dsn = design$clone()
    dsn$setx(!!enquo(x))
    d = dsn$data
    dsn$storename("x", colnames(d))
    
    q_name = quantiles
    qvec = c()
    b = T
    
    for (i in 1:length(quantiles)) {
        if (quantiles[i] >= 1) {
            qvec[i] = 
                dbmax(data = d, var = !!sym(dsn$names$x))
        } else if (quantiles[i] <= 0) {
            qvec[i] = 
                dbmin(data = d, var = !!sym(dsn$names$x))
        } else {
            if (b) {
                N = dsn$getwt()
                nrowdata = db_nrow(d)
                sampN = ceiling(nrowdata^(2/3))
                d = compute(d)
                
                if (!is.null(d$src$con)) {
                  sdata = d %>% 
                    svydb_monet_sampleN(sampN) %>% 
                    tbl_df()
                } else {
                  sdata = d %>% sample_n(sampN) %>% 
                    tbl_df()
                }
                
                sdata = 
                    sdata %>% rename(x = !!sym(dsn$names$x))
                s.surv = svydesign(id = c2f(dsn$id), 
                  st = c2f(dsn$st), weights = c2f(dsn$wt), 
                  data = sdata, nest = T)
                b = F
            }
            notfound = TRUE
            
            while (notfound) {
                q = svyquantile(~x, s.surv, quantiles[i], 
                  alpha = 0.1, ci = TRUE, na.rm = T)
                
                temp_lq = q$CIs[1]
                temp_uq = q$CIs[2]
                
                readIn = d %>% select(x = dsn$names$x, 
                  wt = dsn$wt) %>% filter(x >= temp_lq & 
                  x <= temp_uq)
                readIn_wts = readIn %>% select(wt) %>% 
                  summarise(sum(wt)) %>% pull()
                notRead = d %>% select(x = dsn$names$x, 
                  wt = dsn$wt) %>% filter(x < temp_lq)
                notRead_wts = notRead %>% select(wt) %>% 
                  summarise(sum(wt)) %>% pull()
                
                thold = (N * quantiles[i])
                if ((notRead_wts <= thold) & (thold < 
                  (readIn_wts + notRead_wts))) {
                  qs = readIn %>% arrange(x)
                  qs = qs %>% collect() %>% 
                    mutate(wt2 = cumsum(wt))
                  c_wts = N * quantiles[i] - notRead_wts
                  qs = qs %>% 
                    filter(wt2 >= !!quo(c_wts)) %>% 
                    select(x) %>% slice(1) %>% pull()
                  qvec[i] = qs
                  notfound = F
                } else {
                  if (!is.null(d$src$con)) {
                    sdata = d %>% 
                        svydb_monet_sampleN(sampN) %>% 
                        tbl_df() %>% 
                        rename(x = !!sym(dsn$names$x))
                  } else {
                    sdata = d %>% sample_n(sampN) %>% 
                      tbl_df() %>% 
                      rename(x = !!sym(dsn$names$x))
                  }
                  s.surv = svydesign(id = c2f(dsn$id), 
                    st = c2f(dsn$st), weights = c2f(dsn$wt), 
                    data = sdata, nest = T)
                }
            }
        }
    }
    names(qvec) = as.character(q_name)
    options(oldoptions)
    return(qvec)
}
\end{lstlisting}

%----------------------------------------------------------------
\section{svydbtable}

\begin{lstlisting}
svydbtable = function(formula, design, as.local = F) {
    
    dsn = design$clone()
    dsn$setx(formula)
    d = dsn$data
    d = d %>% filter_all(all_vars(!is.na(.)))
    dsn$storename("all", colnames(d))
    
    ff = all.vars(formula)
    dsn$storename("base", ff[1], force = T)
    
    if (length(ff) == 1) {
        out = d %>% group_by(!!sym(dsn$names$base)) %>% 
            summarise(wt = sum(!!sym(dsn$wt))) %>% 
            arrange(!!sym(dsn$names$base))
        if (as.local == T) {
            out = collect(out)
        }
        return(out)
    }
    
    dsn$storename("by", ff[2], force = T)
    d = d %>% select(dsn$names$all, dsn$wt) %>% 
        dummy_mut(data = ., by = !!sym(dsn$names$by), 
            withBase = T)
    dsn$storename("dummy", colnames(d))
    d = d %>% select(-one_of(dsn$names$by))
    d = compute(d)
    
    if (length(ff) == 2) {
        out = d %>% group_by(!!sym(dsn$names$base)) %>% 
            summarise_at(vars(dsn$names$dummy), 
                funs(sum(. * (!!sym(dsn$wt))))) %>% 
            arrange(!!sym(dsn$names$base))
        if (as.local == T) {
            out = collect(out)
        }
        return(out)
    }
    
    dsn$storename("others", ff[-c(1, 2)], force = T)
    d = db_columnAsCharacter(d, dsn$names$others)
    combnTbl = d %>% select(dsn$names$others) %>% 
        distinct() %>% 
        arrange(!!!syms(dsn$names$others)) %>% 
        collect()
    combnLst = split(combnTbl, seq(1, nrow(combnTbl)))
    
    sTbls = function(by) {
        nname = paste(colnames(by), " = ", by, sep = "",
            collapse = " & ")
        con = gsub(pattern = "=", replacement = "==", 
            nname)
        d = d %>% filter(!!parse_expr(con)) %>% 
            select(-one_of(colnames(by))) %>% 
            group_by(!!sym(dsn$names$base)) %>% 
            summarise_at(vars(dsn$names$dummy), 
                funs(sum(. * (!!sym(dsn$wt))))) %>% 
            arrange(!!sym(dsn$names$base)) %>% 
            list(.)
        names(d) = nname
        d
    }
    out = lapply(combnLst, sTbls) %>% flatten()
    
    if (as.local == T) {
        out = lapply(out, collect)
    }
    
    return(out)
}
\end{lstlisting}

%----------------------------------------------------------------
\section{svydbby}

\begin{lstlisting}
svydbtable = function(formula, design, as.local = F) {
    
    dsn = design$clone()
    dsn$setx(formula)
    d = dsn$data
    d = d %>% filter_all(all_vars(!is.na(.)))
    dsn$storename("all", colnames(d))
    
    ff = all.vars(formula)
    dsn$storename("base", ff[1], force = T)
    
    if (length(ff) == 1) {
        out = d %>% group_by(!!sym(dsn$names$base)) %>% 
            summarise(wt = sum(!!sym(dsn$wt))) %>% 
            arrange(!!sym(dsn$names$base))
        if (as.local == T) {
            out = collect(out)
        }
        return(out)
    }
    
    dsn$storename("by", ff[2], force = T)
    d = d %>% select(dsn$names$all, dsn$wt) %>% 
        dummy_mut(data = ., by = !!sym(dsn$names$by), 
            withBase = T)
    dsn$storename("dummy", colnames(d))
    d = d %>% select(-one_of(dsn$names$by))
    d = compute(d)
    
    if (length(ff) == 2) {
        out = d %>% group_by(!!sym(dsn$names$base)) %>% 
            summarise_at(vars(dsn$names$dummy), 
                funs(sum(. * (!!sym(dsn$wt))))) %>% 
            arrange(!!sym(dsn$names$base))
        if (as.local == T) {
            out = collect(out)
        }
        return(out)
    }
    
    dsn$storename("others", ff[-c(1, 2)], force = T)
    combnTbl = d %>% select(dsn$names$others) %>% 
        distinct() %>% 
        arrange(!!!syms(dsn$names$others)) %>% 
        collect()
    combnLst = split(combnTbl, seq(1, nrow(combnTbl)))
    
    sTbls = function(by) {
        nname = paste(colnames(by), " = ", by, 
            collapse = " & ")
        con = gsub(pattern = "=", replacement = "==", 
            nname)
        d = d %>% filter(!!parse_expr(con)) %>% 
            select(-one_of(colnames(by))) %>% 
            group_by(!!sym(dsn$names$base)) %>% 
            summarise_at(vars(dsn$names$dummy), 
                funs(sum(. * (!!sym(dsn$wt))))) %>% 
            arrange(!!sym(dsn$names$base)) %>% 
            list(.)
        names(d) = nname
        d
    }
    out = lapply(combnLst, sTbls) %>% flatten()
    
    if (as.local == T) {
        out = lapply(out, collect)
    }
    
    return(out)
}
\end{lstlisting}

%----------------------------------------------------------------
\section{svydbrepdesign}

\begin{lstlisting}
makesvydbrepdesign <- R6Class("svydb.repdesign", 
    public = list(dataOg = NULL, data = NULL, 
        vars = NULL, st = NULL, id = NULL, wt = NULL, 
        repwt = NULL, scale = NULL, names = list(), 
        initialize = function(vars = NA, st = NA, 
            id = NA, wt = NA, repwt = NULL, scale, 
            data) {
            
            if (quo_is_null(wt)) {
                stop("Please provide sampling weights")
            } else {
                self$wt = as.character(wt)[2]
            }
            
            if (is.null(repwt)) {
                stop("Please provide replicate weights")
            } else {
                self$repwt = grep(pattern = repwt, 
                  colnames(data), value = T)
            }
            
            if (quo_is_null(st)) {
                data = data %>% mutate(st = 1)
                self$st = "st"
            } else {
                self$st = as.character(st)[2]
            }
            
            if (quo_is_null(id)) {
                data = data %>% mutate(id = row_number())
                self$id = "id"
            } else {
                self$id = as.character(id)[2]
            }
            self$scale = scale
            self$data = data %>% select(everything())
            self$dataOg <<- self$data
        }, setx = function(x) {
            tc = tryCatch(class(x), error = function(e) e)
            
            if ("formula" %in% tc) {
                x = all.vars(x)
                self$data <<- self$data %>% select(!!x, 
                  st = self$st, id = self$id, 
                  self$wt, self$repwt) %>% 
                  filter_all(any_vars(!is.na(.)))
                self$vars <<- x
            } else {
                x = enquo(x)
                self$data <<- self$data %>% select(!!x, 
                  st = self$st, id = self$id, 
                  self$wt, self$repwt) %>% 
                  filter(!is.na(!!x))
                self$vars <<- as.character(x)[2]
            }
            self$names[["logged"]] = c(self$st, 
                self$id, self$wt, self$repwt, 
                "m_h")
        }, addx = function(x) {
            l = enquo(x)
            r = syms(colnames(self$data))
            self$data = self$dataOg %>% select(!!l, 
                !!!r)
        }, getwt = function() {
            self$data %>% select(self$wt) %>% 
                summarise_all(sum) %>% pull()
        }, getmh = function() {
            self$data %>% group_by(!!sym(self$st)) %>% 
                summarise(m_h = n_distinct(!!sym(self$id)))
        }, subset = function(..., logical = T) {
            d = self$clone()
            if (logical == T) {
                d$data = d$data %>% filter(...)
            } else {
                d$data = d$data %>% 
                    filter(!!parse_expr(...))
            }
            return(d)
        }, subset_rows = function(from, to) {
            self$dataSub = self$data %>% db_selectRows(., 
                from = from, to = to)
        }, storename = function(name, obj, force = FALSE) {
            if (force == TRUE) {
                self$names$logged = 
                    self$names$logged[-which(
                        self$names$logged %in% obj)]
            }
            if (!all(obj %in% self$names$logged)) {
                new = setdiff(obj, self$names$logged)
                self$names[[name]] = c(new)
                self$names$logged = c(self$names$logged, 
                  new)
            }
        }, removename = function(name, obj) {
            self$names$logged = 
                self$names$logged[-which(
                    self$names$logged %in% obj)]
            self$names[[name]] = 
                (self$names[[name]])[-which(
                        self$names[[name]] %in% obj)]
        }, print = function() {
            rows = self$data %>% db_nrow()
            txt = sprintf("svydb.repdesign, 
                %s observation(s), %s sets of 
                replicate weights, scale = %s", 
                rows, length(self$repwt), self$scale)
            cat(txt)
}))

svydbrepdesign = function(st = NULL, id = NULL, 
    wt = NULL, repwt = NULL, scale, data) {
    st = enquo(st)
    id = enquo(id)
    wt = enquo(wt)
    
    d = makesvydbrepdesign$new(st = st, id = id, 
        wt = wt, repwt = repwt, scale = scale, 
        data = data)
    d
}
\end{lstlisting}

%----------------------------------------------------------------
\section{svydbreptotal}

\begin{lstlisting}
svydbreptotal = function(x, design, 
            num, return.replicates = F) {
    x = enquo(x)
    
    if (!("svydb.repdesign" %in% class(design))) {
        stop("Please provide a svydb.repdesign")
    }
    
    if (missing(num)) {
        stop("Is x a numeric or categorical variable?, 
            num = T OR num = F?")
    }
    
    dsn = design$clone()
    dsn$setx(!!enquo(x))
    
    d = dsn$data
    
    dsn$storename("x", colnames(d))
    
    if (num == F) {
        d = dummy_mut(d, !!sym(dsn$names$x), withBase = T)
    }
    
    dsn$storename("x", colnames(d))
    
    fullTotTbl = d %>% summarise_at(vars(dsn$names$x), 
        funs(sum(. * (!!sym(dsn$wt))))) %>% collect()
    
    cnt = 1
    getRepTots = function(names, fullTot) {
        replicates = d %>% summarise_at(vars(dsn$repwt), 
            funs(sum((. * !!sym(names))))) %>% 
            collect()
        repTot = replicates %>% summarise_all(funs((. - 
            !!quo(fullTot[cnt]))^2))
        cnt <<- cnt + 1
        if (return.replicates == T) {
            list(replicates = replicates, 
                repVar = db_rowSums(repTot) %>% 
                transmute_all(
                    funs(. * !!quo(dsn$scale))) %>% 
                collect())
        } else {
            list(repVar = db_rowSums(repTot) %>% 
                transmute_all(funs(. * 
                    !!quo(dsn$scale))) %>% 
                collect())
        }
    }
    
    ans = lapply(colnames(fullTotTbl), getRepTots, 
        fullTot = as.vector(t(fullTotTbl)))
    repVar = lapply(ans, function(x) x$repVar) %>% 
        Reduce(rbind, .) %>% pull()
    
    tot = fullTotTbl %>% t() %>% as.vector()
    attr(tot, "var") = repVar
    attr(tot, "statistic") <- "Total"
    attr(tot, "name") = dsn$names$x
    
    if (return.replicates == T) {
        replicates = 
            lapply(ans, function(x) x$replicates %>% 
            collect()) %>% Reduce(rbind, .)
        tot = list(svydbrepstat = tot, 
            replicates = replicates)
    }
    class(tot) = c("svydbrepstat")
    return(tot)
}
\end{lstlisting}

%----------------------------------------------------------------
\section{svydbrepmean}

\begin{lstlisting}
svydbrepmean = function(x, design, num, 
                    return.replicates = F) {
    x = enquo(x)
    
    if (!("svydb.repdesign" %in% class(design))) {
        stop("Please provide a svydb.repdesign")
    }
    
    if (missing(num)) {
        stop("Is x a numeric or categorical variable?, 
            num = T OR num = F?")
    }
    
    dsn = design$clone()
    dsn$setx(!!enquo(x))
    
    d = dsn$data
    
    dsn$storename("x", colnames(d))
    
    if (num == F) {
        d = dummy_mut(d, !!sym(dsn$names$x), withBase = T)
    }
    
    dsn$storename("x", colnames(d))
    
    N = dsn$getwt()
    fullMeanTbl = d %>% summarise_at(vars(dsn$names$x), 
        funs(sum(. * (!!sym(dsn$wt)))/!!quo(N))) %>% 
        collect()
    
    repN = d %>% select(dsn$repwt) %>% 
        summarise_all(sum) %>% 
        collect()
    repN = paste(colnames(repN), "/", repN, 
        collapse = " ; ")
    cnt = 1
    getRepTots = function(names, fullMean) {
        replicates = d %>% summarise_at(vars(dsn$repwt), 
            funs(sum(. * !!sym(names)))) %>% 
            transmute(!!!parse_exprs(repN)) %>% 
            compute()
        repMean = replicates %>% summarise_all(funs((. - 
            !!quo(fullMean[cnt]))^2))
        cnt <<- cnt + 1
        if (return.replicates == T) {
            list(replicates = replicates, 
                repVar = db_rowSums(repMean) %>% 
                transmute_all(funs(. * 
                    !!quo(dsn$scale))) %>% 
                collect())
        } else {
            list(repVar = db_rowSums(repMean) %>% 
                transmute_all(funs(. * 
                    !!quo(dsn$scale))) %>% 
                collect())
        }
    }
    ans = lapply(colnames(fullMeanTbl), getRepTots, 
        fullMean = as.vector(t(fullMeanTbl)))
    repVar = lapply(ans, function(x) x$repVar) %>% 
        Reduce(rbind, .) %>% pull()
    
    means = fullMeanTbl %>% t() %>% as.vector()
    attr(means, "var") = repVar
    attr(means, "statistic") <- "Mean"
    attr(means, "name") = dsn$names$x
    
    if (return.replicates == T) {
        replicates = lapply(ans, function(x) 
                x$replicates %>% 
            collect()) %>% Reduce(rbind, .)
        means = list(svydbrepstat = means, 
            replicates = replicates)
    }
    class(means) = c("svydbrepstat")
    return(means)
}
\end{lstlisting}


%----------------------------------------------------------------
\section{svydbhist}

\begin{lstlisting}
svydbhist = function(x, design, binwidth = NULL, 
    xlab = "x", ylab = "Density", ...) {
    
    if (!("svydb.design" %in% class(design))) {
        stop("Please provide a svydb.design")
    }
    
    dsn = design$clone()
    dsn$setx(!!enquo(x))
    d = dsn$data
    dsn$storename("x", colnames(d))
    d_n = d %>% db_nrow()
    
    x_max = d %>% dbmax(!!sym(dsn$names$x), asNum = T)
    x_min = d %>% dbmin(!!sym(dsn$names$x), asNum = T)
    x_range = x_max - x_min
    
    if (is.null(binwidth)) {
        binwidth = ceiling(log2(d_n) + 1)
        pbreaks = pretty(c(x_min, x_max), n = binwidth, 
            min.n = 1)
    } else {
        pbreaks = seq(from = floor(x_min), 
            to = ceiling(x_max),
            by = binwidth)
    }
    
    d = db_cut2(var = !!sym(dsn$names$x), breaks = pbreaks, 
        data = d) %>% arrange(cut)
    dsn$data = d
    
    props = svydbmean(x = cut, design = dsn, num = F, 
        return.mean = T) %>% collect() %>% t()
    colnames(props) = "Mean"
    
    mids = pbreaks[-length(pbreaks)] + diff(pbreaks)/2
    
    if (length(mids) != nrow(props)) {
        props = tbl_df(props) %>% 
            mutate(uqid = row_number())
        d = left_join(mids %>% tbl_df() %>% 
            mutate(uqid = row_number()), 
            props, by = "uqid")
        d = d %>% mutate(Mean = case_when(is.na(Mean) ~ 
            0, TRUE ~ Mean)) %>% select(-uqid)
    } else {
        d = cbind(mids, props) %>% tbl_df()
    }
    
    colnames(d) = c("x", "y")
    d$y = d$y/diff(pbreaks)
    p = ggplot(d) + geom_col(aes(x, y)) +
        labs(x = dsn$names$x, y = ylab)
    
    print(p)
    
    invisible(p)
}
\end{lstlisting}

%----------------------------------------------------------------
\section{svydbboxplot}

\begin{lstlisting}
svydbboxplot = function(x, groups = NULL, design, 
    varwidth = F, outlier = F, all.outlier = F) {
    
    groups = enquo(groups)
    dsn = design$clone()
    dsn$setx(!!enquo(x))
    d = dsn$data
    dsn$storename("x", colnames(d))
    
    if (quo_is_null(groups)) {
        boxes = svydbquantile(x = !!sym(dsn$names$x), 
            quantile = c(0, 0.25, 0.5, 0.75, 1), 
            design = dsn) %>% t() %>% tbl_df() %>% 
            mutate("")
        ax = c(x = "", y = dsn$names$x)
        colnames(boxes) = c(as.character(letters[1:5]), 
            "x")
    } else {
        group_name = as.character(groups)[2]
        dsn$addx(group_name)
        d = dsn$data
        dsn$storename("groups", colnames(d))
        
        group_levels = distinct(d, 
            !!sym(dsn$names$groups)) %>% 
            collect()
        group_names = paste(colnames(group_levels), 
            pull(group_levels), sep = " ")
        group_names2 = paste(colnames(group_levels), 
            paste("'", pull(group_levels), "'", 
                sep = ""), sep = " ")
        
        f = function(x) {
            svydbquantile(x = !!sym(dsn$names$x), 
                quantile = c(0, 0.25, 0.5, 0.75, 
                  1), design = dsn$subset(x, logical = F))
        }
        
        boxes = sapply(gsub(pattern = " ", 
            replacement = "==",  x = , 
            group_names2), f)
        boxes = t(boxes) %>% tbl_df() %>% 
            bind_cols(group_levels)
        ax = c(x = dsn$names$groups, y = dsn$names$x)
        colnames(boxes) = c(as.character(letters[1:5]), 
            "x")
        boxes$x = as.character(boxes$x)
    }
    
    haveOut = F
    outlsLst = list()
    if (outlier == T) {
        boxes = boxes %>% mutate(outUP = d + 1.5 * 
            (d - b), checkUP = ifelse(outUP < 
            e, T, F), outLow = b - 1.5 * (d - 
            b), checkLow = ifelse(outLow > a, 
            T, F))
        if (any(boxes$checkLow) == T) {
            haveOut = T
            boxes = boxes %>% mutate(a = ifelse(checkLow == 
                T, outLow, a))
            outls = paste(gsub(pattern = " ", 
                replacement = "==", x = group_names2), 
                "&", dsn$names$x, "<", boxes$a, 
                collapse = " | ")
            outlsLow = d %>% 
                filter(!!!parse_exprs(outls)) %>% 
                select(x = dsn$names$groups, 
                    y = dsn$names$x) %>% 
                mutate(x = as.character(x))
            if (all.outlier == F) {
                outlsLow = outlsLow %>% group_by(x) %>% 
                  summarise(y = min(y))
            }
            outlsLow = outlsLow %>% tbl_df()
            outlsLst = c(outlsLst, list(outlsUP))
        }
        
        if (any(boxes$checkUP) == T) {
            haveOut = T
            boxes = boxes %>% mutate(e = ifelse(checkUP == 
                T, outUP, e))
            outls = paste(gsub(pattern = " ", 
                replacement = "==", x = group_names2), 
                "&", dsn$names$x, ">", boxes$e, 
                collapse = " | ")
            outlsUP = d %>% 
                filter(!!!parse_exprs(outls)) %>% 
                select(x = dsn$names$groups,
                    y = dsn$names$x) %>% 
                mutate(x = as.character(x))
            if (all.outlier == F) {
                outlsUP = outlsUP %>% group_by(x) %>% 
                  summarise(y = max(y))
            }
            outlsUP = outlsUP %>% tbl_df()
        }
        outls = bind_rows(outlsUP, outlsLow)
        outls = bind_rows(outlsUP, outlsLow)
    }
    
    p = ggplot(boxes) + labs(x = ax["x"], y = ax["y"])
    
    if (varwidth == T) {
        boxwid = d %>% 
            group_by(!!sym(dsn$names$groups)) %>% 
            summarise(wid = n()) %>% collect()
        p$data = p$data %>% 
            mutate(width = (boxwid$wid/sum(boxwid$wid)))
        p = p + geom_boxplot(aes(x = as.factor(x), 
            ymin = a, lower = b, middle = c, upper = d, 
            ymax = e, width = width), stat = "identity")
        
    } else {
        p = p + geom_boxplot(aes(x = as.factor(x), 
            ymin = a, lower = b, middle = c, upper = d, 
            ymax = e), stat = "identity")
    }
    
    if (haveOut == T) {
        utlsLst = Reduce(rbind, outlsLst)
        p = p + geom_point(data = outls, aes(x = x, 
            y = y))
    }
    
    print(p)
    
    return(p)
}
\end{lstlisting}


%----------------------------------------------------------------
\section{svydbhexbin, svydbhexplot}

\begin{lstlisting}
svydbhcell2xy = function(d) {
    xbins = d$xbins
    xbnds = d$xbnds
    c3 = diff(xbnds)/xbins
    ybnds = d$ybnds
    c4 = (diff(ybnds) * sqrt(3))/(2 * d$shape * 
        xbins)
    jmax = d$dimen[2]
    cell = d$cell - 1
    i = cell%/%jmax
    j = cell%%jmax
    y = c4 * i + ybnds[1]
    x = c3 * ifelse(i%%2 == 0, j, j + 0.5) + xbnds[1]
    
    return(list(x = x, y = y))
}


svydbhbin = function(xy, x, y, xName, yName, cell, 
    cnt, xcm, ycm, size, shape, rx, ry, bnd, n) {
    xmin = rx[1]
    ymin = ry[1]
    xr = rx[2] - xmin
    yr = ry[2] - ymin
    c1 = size/xr
    c2 = size * shape/(yr * sqrt(3))
    
    jinc = floor(bnd[2])
    lat = floor(jinc + 1)
    iinc = floor(2 * jinc)
    lmax = floor(bnd[1] * as.integer(bnd[2]))
    con1 = 0.25
    con2 = 1/3
    
    xy = xy %>% mutate(sx = !!quo(c1), sy = !!quo(c2), 
        xmin = !!quo(xmin), ymin = !!quo(ymin))
    xy = xy %>% mutate(sx = sx * (x - xmin))
    xy = xy %>% mutate(sy = sy * (y - ymin))
    xy = xy %>% mutate(j1 = floor(sx + 0.5), 
        i1 = floor(sy + 0.5))
    xy = xy %>% mutate(dist1 = (sx - j1)^2 + 3 * 
        (sy - i1)^2, iinc = !!quo(iinc), lat = !!quo(lat))
    xy = xy %>% mutate(con1 = !!quo(con1), 
        con2 = !!quo(con2), j2 = floor(sx), 
        i2 = floor(sy))
    xy = xy %>% mutate(con3 = (sx - j2 - 0.5)^2 + 
        3 * (sy - i2 - 0.5)^2)
    xy = xy %>% mutate(L = case_when(dist1 < con1 ~ 
        floor(i1 * iinc + j1 + 1), dist1 > con2 ~ 
        floor(floor(sy) * iinc + floor(as.double(sx)) + 
            lat), TRUE ~ case_when(dist1 <= con3 ~ 
        floor(i1 * iinc + j1 + 1), TRUE ~ floor(i2 * 
        iinc + j2 + lat))))
    
    Lfulltbl = xy %>% select(x, y, L)
    cmsTbl = Lfulltbl %>% group_by(L) %>% 
        summarise(xcm = mean(x), ycm = mean(y)) %>% 
        arrange(L) %>% select(-L) %>% 
        collect()
    Ltbl = xy %>% select(L2 = L, wt) %>% group_by(L2) %>% 
        summarise(cnt = sum(wt))
    xy = xy %>% select(x, y) %>% mutate(L2 = row_number())
    xy = left_join(xy, Ltbl, by = "L2") %>% 
        mutate(cnt = case_when(is.na(cnt) ~ 
            0, TRUE ~ cnt)) %>% 
        arrange(L2)
    cntsTbl = xy %>% rename(cell = L2) %>% 
        mutate(lt1 = case_when(cnt >  0 ~ 1, TRUE ~ 0)) %>% 
        filter(lt1 == 1) %>% 
        select(-lt1) %>% collect()
    
    out = list(cell = cntsTbl$cell, count = cntsTbl$cnt, 
        xcm = cmsTbl$xcm, ycm = cmsTbl$ycm, xbins = size, 
        shape = shape, xbnds = rx, ybnds = ry, 
        dimen = bnd, n = n, ncells = nrow(cntsTbl), 
        xlab = xName, ylab = yName)
    xy = svydbhcell2xy(out)
    out = c(xy, out)
    
    return(out)
}

svydbhexbin = function(formula, design, xbins = 30, 
    shape = 1) {
    
    dsn = design$clone()
    dsn$setx(formula)
    dsn$storename("y", all.vars(formula)[1])
    dsn$storename("x", all.vars(formula)[-1])
    d = dsn$data
    d = d %>% rename(x = !!sym(dsn$names$x)) %>% 
        rename(y = !!sym(dsn$names$y)) %>% 
        rename(wt = !!sym(dsn$wt)) %>% 
        filter(!is.na(x)) %>% filter(!is.na(y))
    d = compute(d)
    
    n = d %>% db_nrow()
    x = d %>% select(x)
    y = d %>% select(y)
    
    xbnds = c(dbmin(d, x), dbmax(d, x))
    ybnds = c(dbmin(d, y), dbmax(d, y))
    
    jmax = floor(xbins + 1.5001)
    c1 = 2 * floor((xbins * shape)/sqrt(3) + 1.5001)
    imax = trunc((jmax * c1 - 1)/jmax + 1)
    lmax = jmax * imax
    
    ans = svydbhbin(xy = d, x = x, y = y, 
        xName = dsn$names$x, yName = dsn$names$y, 
        cell = as.integer(lmax), cnt = as.integer(lmax), 
        xcm = as.integer(lmax), ycm = as.integer(lmax), 
        size = xbins, shape = shape, 
        rx = as.double(xbnds), ry = as.double(ybnds), 
        bnd = as.integer(c(imax, jmax)), n = n)
    
    return(ans)
}

svydbhexplot = function(d, xlab = d$xlab, ylab = d$ylab) {
    pdata = tibble(x = d$x, y = d$y, count = d$count, 
        xcm = d$xcm, ycm = d$ycm)
    
    p = ggplot(pdata) + geom_hex(aes(x = x, y = y, 
        fill = count), 
            color = "black", stat = "identity") + 
        labs(x = xlab, y = ylab) + 
        scale_fill_continuous(trans = "reverse")
    
    print(p)
    
    invisible(p)
}
\end{lstlisting}


%----------------------------------------------------------------
\section{svydbcoplot}

\begin{lstlisting}
svydbcoplot = function(formula, by, design) {
    if (!is_formula(by)) {
        stop("by must be a formula")
    }
    
    y = all.vars(formula)[1]
    x = all.vars(formula)[-1]
    dsn = design$clone()
    
    by_var = all.vars(by)
    by = dsn$data %>% select(!!!syms(by_var)) %>% 
        distinct() %>% arrange(!!sym(by_var[1])) %>% 
        collect()
    by = split(by, seq(nrow(by)))
    
    filterData = function(by, dsn, x, y) {
        dsn = dsn$subset(paste(colnames(by), " == ", 
            by, collapse = " & "), logical = F)
        hb = svydbhexbin(formula, design = dsn)
        if (length(hb$x) | length(hb$y) 
                | length(hb$count) != 0) {
            cbind(tibble(x = hb$x, y = 
                hb$y, count = hb$count),  by)
        }
    }
    
    p = lapply(by, filterData, dsn = dsn) %>% 
        Reduce(rbind, .)
    
    p = ggplot(p) + geom_hex(aes(x = x, y = y, 
        fill = count), 
            color = "black", stat = "identity") + 
        labs(x = x, y = y) + 
        scale_fill_continuous(trans = "reverse") + 
        facet_wrap(by_var, labeller = "label_both")
    
    print(p)
    
    invisible(p)
}
\end{lstlisting}


%----------------------------------------------------------------
\section{Other Functions} \label{appena:otherfunc}

\begin{lstlisting}
dbmin = function(data, var, asNum = T) {
    var = enquo(var)
    data = data %>% ungroup() %>% select(!!var) %>% 
        summarise(min = min(!!var))
    if (asNum == T) {
        data %>% pull
    } else {
        data
    }
}

dbmax = function(data, var, asNum = T) {
    var = enquo(var)
    data = data %>% ungroup() %>% select(!!var) %>% 
        summarise(max = max(!!var))
    if (asNum == T) {
        data %>% pull
    } else {
        data
    }
}

db_nrow = function(data) {
    data %>% ungroup() %>% count() %>% pull()
}

db_dim = function(data) {
    n_rows = data %>% ungroup() %>% count() %>% 
        pull()
    n_cols = ncol(data)
    
    out = c(n_rows, n_cols)
    names(out) = c("rows", "cols")
    out
}

db_view = function(data, num = 1) {
    if (class(data) == "list") {
        View(data[[num]] %>% tbl_df())
    } else {
        View(data %>% tbl_df)
    }
}

db_rowSums_mut = function(data, vars = NULL, 
                            newRowName = "rSum") {
    if (is.null(vars)) {
        s = paste(colnames(data), collapse = " + ")
    } else {
        s = paste(vars, collapse = " + ")
    }
    q = quote(mutate(data, rSum = s))
    
    eval(parse(text = sub("rSum", newRowName, 
        sub("s", s, deparse(q)))))
}

db_rowSums = function(data) {
    cn = colnames(data)
    rs = paste("`", cn, "`", sep = "", collapse = " + ")
    data %>% transmute(rowsum = !!parse_expr(rs))
}

db_cbind = function(x, y) {
    x = x %>% mutate(`___i` = "key")
    y = y %>% mutate(`___i` = "key")
    out = inner_join(x, y, by = "___i", copy = T) %>% 
        select(-`___i`)
    return(out)
}

db_slice = function(data, n) {
    n = enquo(n)
    data %>% mutate(`___i` = row_number()) %>% 
        filter(`___i` <= !!n) %>% select(-`___i`)
}

dummy_mut = function(data, by, withBase = T, 
                            return.level = F) {
    by = enquo(by)
    dum = data %>% distinct(!!by) %>% arrange(!!by) %>% 
        data.frame() %>% na.omit()
    level = as.character(dum[, 1])
    cs = contrasts(as.factor(dum[, 1]))
    by_name = colnames(dum)
    
    if (withBase == T) {
        c1 = c(1, rep(0, nrow(dum) - 1))
        dum = cbind(dum, c1, contrasts(as.factor(dum[, 
            1])))
        name = paste(colnames(dum)[1], as.character(dum[, 
            1]), sep = "_")
        colnames(dum) = c(colnames(dum)[1], name)
    } else {
        dum = cbind(dum, contrasts(as.factor(dum[, 
            1])))
        name = paste(colnames(dum)[1], as.character(dum[, 
            1])[-1], sep = "_")
        colnames(dum) = c(colnames(dum)[1], name)
    }
    
    dum = inner_join(data, dum, by = by_name, 
        copy = T)
    
    if (withBase == F) {
        dum = dum %>% select(-!!by)
    }
    
    if (return.level == T) {
        dum = list(dum = dum, levels = level)
        return(dum)
    } else {
        return(dum)
    }
}

db_cut = function(var, breaks, data) {
    var = enquo(var)
    var_name = as.character(var)[2]
    data = data %>% rename(vars = !!var) %>% 
        filter(!is.na(vars))
    breaks = breaks[-1]
    trues = seq(length(breaks))
    temp_exprs = paste("ifelse(vars", "<= ", breaks, 
        ",", trues, ", _f_)")
    temp_exprs[length(breaks)] = gsub(pattern = "[_]f[_]", 
        replacement = "NA", x = temp_exprs[length(breaks)])
    mut_exprs = temp_exprs[1]
    
    for (i in 2:length(breaks)) {
        mut_exprs = gsub(pattern = "[_]f[_]", 
            replacement = temp_exprs[i], x = mut_exprs)
    }
    
    mut_exprs = parse_expr(mut_exprs)
    data = data %>% mutate(`_cuts_` = !!mut_exprs) %>% 
        rename(`:=`(!!sym(var_name), vars)) %>% 
        rename(cuts = `_cuts_`)
    data
}

db_cut2 = function(var, breaks, right = TRUE, 
    data) {
    var = enquo(var)
    mult = diff(breaks)[1]
    
    if (right == TRUE) {
        data = data %>% mutate(cut = ((!!quo(mult)) * 
            ceiling((!!var)/(!!quo(mult)))) - 
            (!!quo(mult)))
    } else {
        data = data %>% mutate(cut = ((!!quo(mult)) * 
            floor((!!var)/(!!quo(mult)))))
    }
    
    return(data)
}

svydbVar2 = function(x, xleft = 1, xright = 2, 
    st, m_h, data) {
    xleft = x[xleft]
    xright = x[xright]
    m = data
    m = m %>% select(xleft = !!sym(xleft), 
        xright = !!sym(xright), st = st, m_h = m_h)
    m_h = m %>% select(st, m_h) %>% mutate(m_h = 1) %>% 
        group_by(st) %>% summarise(m_h = sum(m_h))
    m = m %>% select(-m_h)
    m = m %>% mutate(ztz = xleft * xright) %>% 
        group_by(st) %>% summarise(ztz = sum(ztz))
    m = left_join(m, m_h, by = "st")
    m = m %>% mutate(scaled = ztz * (m_h/(m_h - 1))) %>% 
        select(scaled) %>% summarise(sum(scaled)) %>% 
        compute(temporary = T) %>% pull()
    return(m)
}
svydbVar = function(x, st, m_h, data) {
    m = data
    m = m %>% select(x = x, st = st, m_h = m_h)
    m_h = m %>% select(st, m_h) %>% mutate(m_h = 1) %>% 
        group_by(st) %>% summarise(m_h = sum(m_h))
    m = m %>% select(-m_h)
    m = m %>% mutate(ztz = x * x) %>% group_by(st) %>% 
        summarise(ztz = sum(ztz))
    m = left_join(m, m_h, by = "st")
    m = m %>% mutate(scaled = ztz * (m_h/(m_h - 1))) %>% 
        select(scaled) %>% summarise(sum(scaled)) %>% 
        compute(temporary = T) %>% pull()
    return(m)
}

c2f = function(x) {
    as.formula(paste("~", x, sep = ""))
}

db_columnAsCharacter = function(x, cols){
  
  checktype =  x %>% select(!!!syms(cols)) %>% 
    head(1) %>% collect %>% lapply(type_sum) %>% 
    unlist()
  
  numCols = checktype[checktype %in% 
                c("int", "dbl")] %>% names()
  
  for(i in 1:length(numCols)){
    x = x %>% 
        mutate(!!quo_name(numCols[1]) := 
            as.character(!!sym(numCols[1])))
  }
  
  x
}

svydb_monet_sampleN = function(data, n) {
    q = paste("SELECT * FROM", data$ops$x, "SAMPLE", 
        n)
    dbGetQuery(data$src$con, q)
}

as.data.frame.svydbstat = function(x) {
    ans = cbind(coef(x), SE(x))
    colnames(ans) = c(attr(x, "statistic"), "SE")
    ans
}

print.svydbstat = function(xx, ...) {
    v <- attr(xx, "var")
    m = cbind(xx, sqrt(v))
    colnames(m) = c(attr(xx, "statistic"), "SE")
    printCoefmat(m)
}

coef.svydbstat = function(object, ...) {
    attr(object, "statistic") = NULL
    attr(object, "name") = NULL
    attr(object, "var") = NULL
    unclass(object) %>% t() %>% as.vector()
}

SE.svydbstat = function(x, ...) {
    s = attr(x, "var") %>% sqrt()
    names(s) = attr(x, "name")
    return(s)
}

print.svydbrepstat = function(xx, ...) {
    if (is.list(xx)) {
        xx = xx$svydbrepstat
    }
    v <- attr(xx, "var")
    m = cbind(xx, sqrt(v))
    colnames(m) = c(attr(xx, "statistic"), "SE")
    rownames(m) = attr(xx, "name")
    printCoefmat(m)
}

coef.svydbrepstat = function(object, ...) {
    if (is.list(object)) {
        object = object$svydbrepstat
    }
    attr(object, "statistic") = NULL
    attr(object, "name") = NULL
    attr(object, "var") = NULL
    unclass(object) %>% t() %>% as.vector()
}

SE.svydbrepstat = function(x, ...) {
    if (is.list(x)) {
        x = x$svydbrepstat
    }
    s = attr(x, "var") %>% sqrt()
    names(s) = attr(x, "name")
    return(s)
}

print.svydblm = function(x, digits = 
    max(3L, getOption("digits") - 3L), ...) {
    print(x$design)
    cat("\nSurvey design:\n")
    print(x$design$call)
    cat("\nCall:\n", paste(deparse(x$call), sep = "\n", 
        collapse = "\n"), "\n\n", sep = "")
    if (length(coef(x))) {
        cat("Coefficients:\n")
        print.default(format(coef(x), digits = digits), 
            print.gap = 2L, quote = FALSE)
    } else cat("No coefficients\n")
    cat("\n")
    invisible(x)
}
summary.svydblm = function(object) {
    df.r = object$df.residual
    coef.p = coef(object)
    covmat = vcov(object)
    dimnames(covmat) = list(colnames(coef.p), 
        colnames(coef.p))
    var.cf = diag(covmat)
    s.err = sqrt(var.cf)
    tvalue = coef.p/s.err
    dn = c("Estimate", "Std. Error")
    pvalue <- 2 * pt(-abs(tvalue), df.r)
    coef.table <- rbind(coef.p, t(as.matrix(s.err)), 
        tvalue, pvalue) %>% t()
    dimnames(coef.table) <- list(colnames(coef.p), 
        c(dn, "t value", "Pr(>|t|)"))
    ans = list(df.residual = df.r, 
        coefficients = coef.table, 
        cov.unscaled = covmat, cov.scaled = covmat, 
        call = object$call, design = object$design)
    class(ans) <- c("summary.svydblm", "summary.glm")
    return(ans)
}
print.summary.svydblm = function(x, digits = max(3, 
    getOption("digits") - 3), 
    signif.stars = getOption("show.signif.stars"), ...) {
    cat("\nCall:\n")
    cat(paste(deparse(x$call), sep = "\n", collapse = "\n"), 
        "\n\n", sep = "")
    cat("Survey design:\n")
    print(x$design$call)
    cat("\nCoefficients:\n")
    coefs <- x$coefficients
    if (!is.null(aliased <- is.na(x$coefficients[, 
        1])) && any(aliased)) {
        cn <- names(aliased)
        coefs <- matrix(NA, length(aliased), 4, 
            dimnames = list(cn, colnames(coefs)))
        coefs[!aliased, ] <- x$coefficients
    }
    printCoefmat(coefs, digits = digits, 
        signif.stars = signif.stars, 
        na.print = "NA", ...)
    invisible(x)
}

predict.svydblm = function(object, newdata = NULL, 
    ...) {
    tt = delete.response(object$terms)
    mf = model.frame(tt, data = newdata, 
        xlev = object$design$levels)
    mm = model.matrix(tt, mf)
    eta = drop(mm %*% as.vector(coef(object)))
    attr(eta, "var") = drop(rowSums((mm %*% vcov(object)) * 
        mm))
    attr(eta, "statistic") = "link"
    class(eta) <- "svydbstat"
    eta
}

vcov.svydblm = function(x, ...) {
    x$cov.unscaled
}
\end{lstlisting}
%\chapter{\textbf{shiny} app} \label{AppendixB}

\section{Joining module}
\subsection{UI}
\begin{lstlisting}
library(shiny)
library(dataAnim)
library(shinyjs)
library(shinyalert)
library(V8)
shinyUI(fluidPage(
  useShinyalert(),
  useShinyjs(),
  extendShinyjs(text =
                  "shinyjs.clearpage = function(){
                    d3.select('svg').remove()}"),
  extendShinyjs(text =
                  "shinyjs.begin = function(){if(d3
                    .select('svg').node() === null)
                            {d3.select('#mypanel')
                                .append('svg')}}"),
  titlePanel("Joining Animation"),
  sidebarLayout(
    sidebarPanel(
      fileInput("xtbl_upload", "Choose CSV File for Table 1",
                multiple = FALSE,
                accept = c("text/csv",
                           "text/comma-separated-values,
                            text/plain", ".csv")),
      fileInput("ytbl_upload", "Choose CSV File for Table 2",
                multiple = FALSE,
                accept = c("text/csv",
                           "text/comma-separated-values,
                            text/plain", ".csv")),
      selectInput("jointype_input", "Join Type", c("Left", 
        "Inner", "Complete"),
                  selected = "Left", multiple = FALSE),
      selectInput("join_var", "Variable to join by", NULL, 
        multiple = FALSE), sliderInput("speed_sld",
                  "Animation Speed",
                  min = 1,
                  max = 5,
                  value = 1),
      checkboxInput("msg_chk", "Show annotations", value = F),
      fluidRow(actionButton(inputId = "go_btn", label = "Go!"),
               actionButton(inputId = "clear_btn", 
                label = "Clear")), br(),
      fluidRow(column(8), downloadButton(
        "download_btn", "Download Animation")),
      width = 3
    ),
    mainPanel(
      join_animOutput("animpanel0")
    )
  )
))
\end{lstlisting}

\subsection{Server}
\begin{lstlisting}
library(shiny)
get_id = function(txt, cnt){
  txt = gsub("[0-9]", "", txt)
  return(paste0(txt, cnt))
}
shinyServer(function(input, output, session) {
x_tbl = reactiveValues(data = NULL)
  y_tbl = reactiveValues(data = NULL)
  store = reactiveValues(container_id = "animpanel0", 
    cnt = 1, anim = NULL)
  observeEvent(input$go_btn, {
    shinyjs::disable("go_btn")
    output$animpanel0<- join_animRender({
      if(is.null(x_tbl$data) | is.null(y_tbl$data)) {
        return()
      }
      store$anim = dataAnim::join_anim(join_type = 
        tolower(input$jointype_input), 
            speed = input$speed_sld, x = isolate(x_tbl$data),
                 y = isolate(y_tbl$data), by = input$join_var,
                    show_msg = input$msg_chk)
      return(store$anim)
    })
  })
  observeEvent(c(input$xtbl_upload, input$ytbl_upload), {
    in_x = input$xtbl_upload
    in_y = input$ytbl_upload
    if(is.null(in_x) | is.null(in_y)) {
      return(NULL)
    }
    x_tbl$data = read.csv(in_x$datapath, header = T,
                          stringsAsFactors = F)
    y_tbl$data = read.csv(in_y$datapath, header = T,
                          stringsAsFactors = F)
    updateSelectInput(session, inputId = "join_var",
                      choices = intersect(colnames(x_tbl$data),
                                colnames(y_tbl$data)))
  })
  observeEvent(input$clear_btn, {
    js$clear_page()
    shinyjs::enable("go_btn")
  })
  output$download_btn = downloadHandler(
    filename = function() {
      paste0("animation.html")
    },
    content = function(file) {
      if(is.null(store$anim)) {
        shinyalert::shinyalert("Download Failed", 
            "Please check if your animation is loaded.",
                type = "error")
        return()
      }
      shinyalert::shinyalert("Download Success", 
        "Go check out your animation!", type = "success")
      htmlwidgets::saveWidget(store$anim, file = 
        "animation.html", selfcontained = TRUE)
      file.copy('animation.html', file)
    }
  )
})

\end{lstlisting}

\section{Reshaping module}

\subsection{UI}

\begin{lstlisting}
library(shiny)
library(shinyjs)
library(dataAnim)
shinyUI(fluidPage(
  sidebarLayout(
    sidebarPanel(
      tabsetPanel(
        
        tabPanel("Data", 
                 br(),
                 fileInput("data_upload", "Upload CSV File",
                           multiple = FALSE,
                           accept = c("text/csv",
                             "text/comma-separated-values",
                                ".csv"))
        ),
        tabPanel("Spread", 
                br(),
                selectInput("s_key", "Select the column to 
                spread out to multiple columns (key)", ""),
                selectInput("s_value", "Select the column 
                with the values to be put in these 
                columns (value)", ""),
                fluidRow(actionButton(inputId = "s_go_btn", 
                label = "Go!"),
                    actionButton(inputId = "s_clear_btn", 
                        label = "Clear")), br(),
                 fluidRow(column(8), downloadButton(
                    "g_download_btn", "Download Animation"))),
        tabPanel("Gather",
                 br(),
                 textInput("g_key", "Name the new column 
                    containing the old column names (Key)"),
                 textInput("g_value", "Name the new column 
                    containing the old column values (Value)"),
                 selectInput("g_col", 
                    "Select columns to gather together", ""),
                 fluidRow(actionButton(inputId = "g_go_btn", 
                    label = "Go!"),
                          actionButton(inputId = "g_clear_btn", 
                            label = "Clear")),
                 br(),
                 fluidRow(column(8), downloadButton(
                    "g_download_btn", 
                    "Download Animation")))
      )
    ),
    mainPanel(
      spread_animOutput("spread_panel"),
      gather_animOutput("gather_panel")
    )
  )
))

\end{lstlisting}

\subsection{Server}

\begin{lstlisting}
library(shiny)
library(shinyjs)
library(dataAnim)
shinyServer(function(input, output) {
  observeEvent(input$data_upload, {
    in_x = input$data_upload
    x_tbl$data = read.csv(in_x$datapath, header = T,
                          stringsAsFactorsrshiny = F)
    updateSelectInput(session, inputId = "g_col", choices = 
        colnames(x_tbl))
    updateSelectInput(session, inputId = "s_key", choices = 
        colnames(x_tbl))
    updateSelectInput(session, inputId = "s_value", choices = 
        colnames(x_tbl))
  })
  observeEvent(input$g_clear_btn, {
    # js$clearpage()
  })
  observeEvent(input$s_clear_btn, {
    # js$clearpage()
  })
  dl = function(file) {
    browser()
    if(is.null(store$anim)) {
      shinyalert::shinyalert("Download Failed", 
        "Please check if your animation is loaded.", 
            type = "error")
      return()
    }
    shinyalert::shinyalert("Download Success", 
        "Go check out your animation!", type = "success")
    htmlwidgets::saveWidget(store$anim, file = 
        "animation.html", selfcontained = TRUE)
    file.copy('animation.html', file)
  }
  output$g_download_btn = downloadHandler(
    filename = function() {
      paste0("animation.html")
    },
    content = dl
  )
  output$s_download_btn = downloadHandler(
    filename = function() {
      paste0("animation.html")
    },
    content = dl
  )
})

\end{lstlisting}

%----------------------------------------------------------------------------------------
%	BIBLIOGRAPHY
%----------------------------------------------------------------------------------------

\printbibliography[heading=bibintoc]

%----------------------------------------------------------------------------------------

\end{document}  
