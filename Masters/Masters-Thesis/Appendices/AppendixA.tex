% Appendix Template

\chapter{\textbf{dataAnim} Package in \textsf{R}} \label{AppendixA}

\section{Joining Module}

\subsection{\texttt{join\_anim}}
\begin{lstlisting}[language=R]
#' Joining Animation
#'
#' @param join_type Type of join.
#' @param x Table x, to pass to join.
#' @param y Table y, to pass to join.
#' @param by A character value of the variable to join by.
#' @param speed Speed of the animation.
#' @param width Width of the animation frame.
#' @param height Height of the animation frame.
#' @param show_msg A logical value indicating whether to 
    show instructional messages in the animation.
#' @description
#' Function to create joining animations.
#' @examples
#' data(datoy1)
#' join_anim(join_type = "left", speed = 1, x = datoy1$x, 
    y = datoy1$y, by = "Name", show_msg = T)
#' myanim = join_anim(join_type = "inner", speed = 1, 
    x = datoy1$x, y = datoy1$y, by = "Name", show_msg = F)
#' htmlwidgets::saveWidget(myanim)
#' @author Charco Hui
#' @import htmlwidgets
#'
#' @export
join_anim <- function(join_type = "left", x, y, by, 
    speed = 1, width = NULL, height = NULL,
                      show_msg = F) {
  if(nrow(x) >= 10 || nrow(y) >= 10) {
    stop("Dataset cannot have more than 10 rows.")
  }
  if(ncol(x) >= 5 || ncol(y) >= 5) {
    stop("Dataset cannot have more than 5 columns.")
  }
  data = list(data = process_join(x = x, y = y, key = by, 
    complete_action = TRUE, show_msg = show_msg, 
        join_type = join_type, asJSON = TRUE),
              speed = speed, join_type = join_type)
  out = htmlwidgets::createWidget(
    name = "join_anim",
    data,
    width = width,
    height = height,
    package = 'dataAnim'
  )
  return(out)
}
#' @export
join_animOutput <- function(outputId, width = "100%", 
    height = "1000") {
  shinyWidgetOutput(outputId, "join_anim", width, height,
    package = "dataAnim")
}
#' @export
join_animRender <- function(expr, env = parent.frame(), 
    quoted = FALSE) {
  if (!quoted) { expr <- substitute(expr) } # force quoted
  shinyRenderWidget(expr, join_animOutput, env, 
    quoted = TRUE)
}

\end{lstlisting}

\subsection{Helper functions for joins}

\begin{lstlisting}
initial_prep = function(from_tbl, to_tbl, key, join_type, 
    show_msg) {
  from_ind = which(colnames(from_tbl) == key)
  to_ind = which(colnames(to_tbl) == key)
  initial = sapply(to_tbl[,to_ind], FUN = function(x) 
    which(from_tbl[,from_ind] == x))
  initial = lapply(initial, FUN = function(x) {
    if(length(x) == 0) {
      return(-1)
    } else {
      return(x)
    }
  })
  when = lapply(initial, getwhen) %>% Reduce(rbind, .) 
    %>% as.vector()
  msg = lapply(initial, function(x) getmsg(x, join_type)) 
    %>% Reduce(rbind, .) %>%
    as.vector()
  msg = data.frame(name = names(initial), 
    msg = as.character(msg))
  msg = msg %>% group_by(msg) %>% mutate(uq = row_number()) 
    %>%ungroup() %>% mutate(msg = as.character(msg)) %>%
    mutate(msg = ifelse(uq == 1, msg, NA)) %>% select(-uq)
  msg = msg %>% rowwise() %>% transmute(msg = gsub("_val_", 
    name, msg)) %>% pull(msg)
  initial = tibble(row = seq(length(initial)), 
    dest = initial)
  if(isTRUE(show_msg)) {
    initial = initial %>% mutate(msg = msg, when = when)
  }
  initial = initial %>% unnest() %>% split(.$row)
  return(initial)
}
find_key_col = function(obj, key, minus1 = FALSE){
  ans = which(colnames(obj) == key)
  if(minus1 == TRUE){
    return(ans - 1)
  }else{
    return(ans)
  }
}
xy_loc = function(from_tbl, to_tbl, result_tbl, key, 
    minus1 = F) {
  result_tbl = result_tbl %>% mutate(stop = row_number())
  from_cn = colnames(from_tbl)
  from_tbl = from_tbl %>% mutate(start = row_number())
  out = left_join(result_tbl, from_tbl, by = from_cn) %>%
    select(start, stop) %>% na.omit() %>% arrange(stop)
  return(out)
}
xy_loc2 = function(from_tbl, to_tbl, result_tbl, key, 
    minus1 = F) {
  result_tbl = result_tbl %>% 
    group_by(!!!syms(colnames(to_tbl))) %>%
    mutate(addrow = row_number()) %>% ungroup()
  from_cn = colnames(from_tbl)
  from_tbl = from_tbl %>% mutate(start = row_number())
  to_cn = colnames(to_tbl)
  to_tbl = to_tbl %>% mutate(split = row_number())
  result_tbl = left_join(result_tbl, from_tbl, by = from_cn) 
    %>% left_join(., to_tbl, by = to_cn) %>%
    mutate(stop = row_number())
  if(minus1 == T) {
    result_tbl = result_tbl %>%
      mutate(start = start - 1, stop = stop - 1)
  }
  out = result_tbl %>%
    select(split, start, stop, addrow) %>%
    na.omit() %>% 
    split(x = ., f = .$split)
  out = lapply(out, function(xx) xx %>% select(-split))
  return(out)
}
getmsg = function(x, join_type) {
  na_msg = switch(join_type, inner = "No match for _val_. 
    \nDelete", "No match for _val_")
  if(length(x) > 1) {
    return(sprintf("%s matches found for \n_val_", 
        length(x)))
  }else if(x == -1) {
    return(na_msg)
  }else if(length(x) == 1) {
    return("Matching _val_")
  }
}
getwhen = function(x) {
  if(length(x) > 1) {
    return("after")
  }else if(x == -1) {
    return("after")
  }else if(length(x) == 1) {
    return("before")
  }
}
\end{lstlisting}

\subsection{Back end code for joining}

\begin{lstlisting}
process_join = function(x, y, key, height = 2, width = 5, 
    svg_width = 1920, svg_height = 1080,
                        complete_action = TRUE, 
                            join_type = "left", show_msg, 
                                asJSON = FALSE, ...){
  xy = left_join(x, y, by = key)
  temp = list(x = include_cn(x),
              y = include_cn(y),
              xy = include_cn(xy),
              height = height,
              initial_prep = initial_prep(y ,x, key = key, 
                join_type = join_type, show_msg = show_msg),
              x_key_col = find_key_col(x, key = key),
              y_key_col = find_key_col(y, key = key),
              svg_width = svg_width, svg_height = svg_height)
  if(isTRUE(complete_action)) {
    row2move = temp$initial_prep %>% Reduce(rbind, .) %>%
      filter(dest != -1) %>% pull(dest) %>%
      unique()
    row2move = outersect(1:nrow(y), row2move)
    col_ind = sapply(colnames(temp$y), function(x){
      which(colnames(temp$xy) == x)
    })
    temp = c(temp, list(com_act = list(row2move = row2move, 
        col_ind = col_ind)))
  }
  temp = c(temp,
          list(x_w = col_max_width(temp$x, width = width),
                y_w = col_max_width(temp$y, width = width)))
  temp = c(temp,
          list(x_cord_x = c(0, 
          cumsum(temp$x_w[-length(temp$x_w)])),
                x_cord_y = c(0, 
                cumsum(temp$y_w[-length(temp$y_w)]))))
  if(asJSON == TRUE){
    return(jsonlite::toJSON(temp))
  }else{
    return(temp)
  }
}

\end{lstlisting}

\subsection{\textbf{htmlwidgets} \textsf{JavaScript} code for \texttt{join\_anim} in \textsf{R}}
\begin{lstlisting}
HTMLWidgets.widget({

  name: 'join_anim',

  type: 'output',

  factory: function (el, width, height) {
    let svg_width = width * 0.8;
    let svg_height = svg_width / 1.6;
    return {
      renderValue: function (x) {
        let data = x.data;
        let speed = x.speed;
        let join_type = x.join_type;
        let xtbl_width = arr_sum(data.x_w[0]);
        let ytbl_width = arr_sum(data.y_w[0]);
        let xy_width = xtbl_width + ytbl_width;
        let cell_height = data.height[0];
        let xscale_num = (xy_width + 1.5 * ytbl_width) / 0.9;
        let yscale_num = cell_height * (data.x.length + 
            data.y.length - 1) / 0.9;
        d3.select(el)
          .append("svg")
          .attr("width", svg_width)
          .attr("height", svg_height);

        let x_scale =
          d3.scaleLinear()
          .domain([0, xscale_num])
          .range([0, svg_width]);
        let y_scale =
          d3.scaleLinear()
          .domain([0, yscale_num])
          .range([0, svg_height]);
        let height = y_scale(cell_height);

        let xtbl_start = {
          x: svg_width * 0.05,
          y: svg_height * 0.05
        };
        let ytbl_start = {
          x: svg_width - svg_width * 0.05 - 
            x_scale(ytbl_width),
          y: svg_height * 0.05
        };

        let x_width = arr_scale(data.x_w[0], x_scale);
        let y_width = arr_scale(data.y_w[0], x_scale);
        let x_cord_x = arr_scale(data.x_cord_x, x_scale);
        let x_cord_y = arr_scale(data.x_cord_y, x_scale);
        let y_cord_x = Array.from(Array(data.x.length), 
            (d, i) => i * height);
        let y_cord_y = Array.from(Array(data.y.length), 
            (d, i) => i * height);

        x_rect_cord = draw_table(data.x, xtbl_start["x"], 
            xtbl_start["y"], x_cord_x,
          x_width, height, "x", parseInt(data.x_key_col));
        y_rect_cord = draw_table(data.y, ytbl_start["x"], 
            ytbl_start["y"], x_cord_y,
          y_width, height, "y", parseInt(data.y_key_col));

        let iso_action = join_type;
        if (join_type == "complete") {
          join_type = "left";
        }

        keycol_anim("x", "y", speed, start_time = 0)
          .on("end", function () {
            join_anim(data.initial_prep, speed, join_type, 
                true) .on("end", () => {
                join_finalcheck(data.initial_prep)
                  .on("end", () => {
                    if (iso_action === "complete") {
                      comjoin_final(data.com_act, speed)
                        .on("end", () => {
                          iso_tbl(join_type, height, speed, 0);
                        });
                    } else {
                      iso_tbl(join_type, height, speed, 0);
                    }
                  })
              })
          })
      },
      resize: function (width, height) {}
    };
  }
});
\end{lstlisting}

\section{Reshaping module}

\subsection{\texttt{spread\_anim}}

\begin{lstlisting}
#' Spread Animation
#'
#' @param key Column used to spread out to multiple columns.
#' @param value Column containing values of the key.
#' @param data Data to pass into the function.
#' @param speed Speed of the animation.
#' @param width Width of the animation frame in pixels.
#' @param height Height of the animation frame in pixels.
#' @description
#' Long to Wide transformation.
#' @examples
#' data(toyda_long)
#' spread_anim(key = "Subject", value = "Score", 
    data = toyda_long)
#' myanim = spread_anim(key = "Subject", value = "Score", 
    data = toyda_long)
#' htmlwidgets::saveWidget(myanim, file = "myanim.html")
#' @author Charco Hui
#' @import htmlwidgets
#'
#' @export
spread_anim <- function(key, value, data, speed = 1, 
    width = NULL, height = NULL) {
  if(ncol(data) > 3) {
    stop("Only 3 columns are supported at the moment")
  }
  data = list(data = process_spread(key = key, value = value, 
    data = data, asJSON = TRUE),
              speed = speed)
  out = htmlwidgets::createWidget(
    name = "spread_anim",
    data,
    width = width,
    height = height,
    package = 'dataAnim'
  )
  return(out)
}
#' @export
spread_animOutput <- function(outputId, width = "100%", 
    height = "1000") {
  shinyWidgetOutput(outputId, "spread_anim", width, height, 
    package = "dataAnim")
}
#' @export
spread_animRender <- function(expr, env = parent.frame(), 
    quoted = FALSE) {
  if (!quoted) { expr <- substitute(expr) } 
  shinyRenderWidget(expr, spread_animOutput, env, 
    quoted = TRUE)
}

\end{lstlisting}

\subsection{\texttt{gather\_anim}}

\begin{lstlisting}
#' Gather Animation
#'
#' @param key New column name to contain old columns names.
#' @param value New column name to contain old column valuess.
#' @param col Columns in the original data to transform on.
#' @param data Data to pass into the function.
#' @param speed Speed of the animation.
#' @param width Width of the animation frame in pixels.
#' @param height Height of the animation frame in pixels.
#' @description
#' Wide to long transformation.
#' @examples
#' data(toyda_wide)
#' gather_anim(key = "Subject", value = "Score", 
    col = c("English", "Maths"), data = toyda_wide)
#' myanim = gather_anim(key = "Subject", value = "Score", 
    col = c("English", "Maths"), data = toyda_wide)
#' htmlwidgets::saveWidget(myanim, file = "myanim.html")
#' @author Charco Hui
#' @import htmlwidgets
#'
#' @export
gather_anim <- function(key, value, data, col, speed = 1, 
    width = NULL, height = NULL) {
  if(ncol(data) > 3) {
    stop("Only 3 columns are supported at the moment")
  }
  data = list(data = process_gather(key = key, 
    value = value, col = col, data = data, 
        asJSON = TRUE), speed = speed)
  out = htmlwidgets::createWidget(
    name = "gather_anim",
    data,
    width = width,
    height = height,
    package = 'dataAnim'
  )
  return(out)
}
#' @export
gather_animOutput <- function(outputId, width = "100%",
    height = "1000") {
  shinyWidgetOutput(outputId, "gather_anim", width, 
    height, package = "dataAnim")
}
#' @export
gather_animRender <- function(expr, env = parent.frame(), 
    quoted = FALSE) {
  if (!quoted) { expr <- substitute(expr) }
  shinyRenderWidget(expr, spread_animOutput, env, 
    quoted = TRUE)
}
\end{lstlisting}

\subsection{Back end code for reshaping}

\begin{lstlisting}
process_spread = function(key, value, data, height = 2, 
    width = 5, svg_width = 1920,
                          svg_height = 1080, show_msg = T, 
                            asJSON = FALSE) {
  if(ncol(data) != 3) {
    stop("Inputted dataset must have 3 columns for 
        the animation to work")
  } else if(ncol(data) > 3) {
    stop("Spread animation currently only supports dataset 
        with 3 columns")
  } else if(ncol(data) < 3) {
    stop("This data set is not possible to transform to long")
  }
  cn = colnames(data)
  key_ind = which(cn == key)
  value_ind = which(cn == value)
  pivot_ind = (1:ncol(data))[-c(key_ind, value_ind)]
  data = data %>% arrange(!!sym(key), !!sym(cn[pivot_ind]))
  result = data %>% spread(key = key, value = value)
  na_pos = which(is.na(result), arr.ind = TRUE)
  result[is.na(result)] = ""
  key_seq = data %>% pull(!!sym(key)) %>% as.factor()
  key_seq = cumsum(tabulate(key_seq))
  key_seq = data.frame(start = c(1, 
    key_seq[-length(key_seq)] + 1), stop = key_seq)
  split_data = apply(key_seq, 1, function(x) data[x[1]:x[2],])
  key_rowseq = lapply(split_data, function(x) {
    inner_join(result %>% select(pivot_ind) %>% 
        mutate(`__rn` = row_number()),
              x, by = "Name") %>% pull(`__rn`)
  })
  pivot_rows = which(!duplicated(data[,pivot_ind]))
  out = list(key_ind = key_ind, value_ind = value_ind, 
    original = include_cn(data),
             height = height, svg_width = svg_width, 
                pivot_ind = pivot_ind, key_seq = key_seq,
             svg_height = svg_height, 
                result = include_cn(result), key = key, 
                    value = value,key_rowseq = key_rowseq, 
                        pivot_rows = pivot_rows, 
                pivot = cn[pivot_ind], na_pos = na_pos)
  out = c(out, list(rslt_cn = as.vector(colnames(out$result)),
                    og_w = col_max_width(out$original, 
                        width = width),
                    rslt_w = col_max_width(out$result, 
                        width = width)))
  out = c(out, list(og_cord_x = c(0, 
    cumsum(out$og_w[-length(out$og_w)])),rslt_cord_x = 
        c(0, cumsum(out$rslt_w[-length(out$rslt_w)]))))
  if(isTRUE(asJSON)) {
    return(jsonlite::toJSON(out))
  } else {
    return(out)
  }
}
process_gather = function(key, value, col, data, height = 2, 
    width = 5, svg_width = 1920,
                          svg_height = 1080, show_msg = T, 
                            asJSON = FALSE) {
  if(missing(col)) {
    stop("Columns to keep must be specified.")
  }
  cn = colnames(data)
  pivot_ind = which(!(cn %in% col))
  col_ind = which(cn %in% col)
  result = data %>% gather(key = !!sym(key), 
    value = !!sym(value), col)
  cn2 = colnames(result)
  key_ind = which(cn2 == key)
  value_ind = which(cn2 == value)
  out = list(original = include_cn(data), height = height, 
    svg_height = svg_height, svg_width = svg_width,
        result = include_cn(result), pivot_ind = pivot_ind,
             col_ind = col_ind, key = key, value = value, 
                col = col, value_ind = value_ind,
             key_ind = key_ind)
  out = c(out, list(rslt_cn = as.vector(colnames(out$result)),
                    og_w = col_max_width(out$original, 
                        width = width),
                    rslt_w = col_max_width(out$result, 
                        width = width)))
  out = c(out, list(og_cord_x = c(0, 
    cumsum(out$og_w[-length(out$og_w)])),
                    rslt_cord_x = c(0, 
                        cumsum(out$rslt_w
                            [-length(out$rslt_w)]))))
  if(isTRUE(asJSON)) {
    return(jsonlite::toJSON(out))
  } else {
    return(out)
  }
}
\end{lstlisting}

\subsection{\textbf{htmlwidgets} \textsf{JavaScript} code for \texttt{spread\_anim} \textbf{R}}

\begin{lstlisting}
HTMLWidgets.widget({
    name: 'spread_anim',
    type: 'output',
    factory: function (el, width, height) {
      let svg_width = width * 0.8;
      let svg_height = svg_width / 1.6;
      return {
        renderValue: function (x) {
          let data = x.data;
          svg_div = "animpanel0";
          let otbl_width = arr_sum(data.og_w[0]);
          let rtbl_width = arr_sum(data.rslt_w[0]);
          let or_width = otbl_width + rtbl_width;
          let cell_height = data.height[0];
          let xscale_num = (or_width + 1.5 * rtbl_width) / 0.9;
          let yscale_num = cell_height * (data.original.length + 
            data.original.length - 1) / 0.9;
          el = d3.select("body")
            .append("div")
            .attr("id", svg_div);
          el.append("svg")
            .attr("width", svg_width)
            .attr("height", svg_height);
          let x_scale =
            d3.scaleLinear()
            .domain([0, xscale_num])
            .range([0, svg_width]);
          let y_scale =
            d3.scaleLinear()
            .domain([0, yscale_num])
            .range([0, svg_height]);
          let height = y_scale(cell_height);
          let otbl_start = {
            x: svg_width * 0.05,
            y: svg_height * 0.25
          };
          let ptxt_start = {
            x: svg_width * 0.015,
            y: svg_height * 0.05
          };
          let o_width = arr_scale(data.og_w[0], x_scale);
          let x_cord_o = arr_scale(data.og_cord_x, 
            x_scale);
          let rslt_cord_x = arr_scale(data.rslt_cord_x, x_scale);
          let r_width = arr_scale(data.rslt_w[0], x_scale);
          let r_tbl_tl = {
            x: svg_width - svg_width * 0.05 - arr_sum(r_width),
            y: otbl_start["y"]
          };
          let key_rect_w = x_scale(data.rslt_w[0]
            [data.key_ind[0] - 1]);
          let value_rect_w = x_scale(data.rslt_w[0]
            [data.value_ind[0] - 1]);
          let tbl_mid_xy = {
            x: svg_width / 2,
            y: svg_height / 2
          };
          draw_table_rectonly(data.result, r_tbl_tl["x"], 
            r_tbl_tl["y"], rslt_cord_x,
            r_width, height, "r", parseInt(data.pivot_ind), 
                false);
          d3.selectAll(".r_rows")
            .style("opacity", 0);
          input_text_node = d3.select(".r_rows").
            selectAll(".r_cols").nodes();
          input_text_node.forEach(function (d, i) {
            cur_rect = d3.select(input_text_node[i])
              .select("rect");
            d3.select(input_text_node[i])
              .append("text")
              .attr("x", parseFloat(cur_rect.attr("x")) + 
                parseFloat(cur_rect.
                    attr("width") / 2))
              .attr("y", parseFloat(cur_rect.attr("y")) + 
                parseFloat(cur_rect.
                    attr("height") / 2))
              .style("font-size", height * 0.5)
              .text(data.rslt_cn[i]);
          })
          x_rect_cord = draw_table(data.original, 
            otbl_start["x"], otbl_start["y"], x_cord_o,
            o_width, height, "o", parseInt(data.pivot_ind));
          let all_o_rows = d3.selectAll(".o_rows")
            .nodes();
          let delay_time = 2000;
          let msg = true;
          let speed = data.speed;
          let new_col_cnt = 2;
          prepare_spread(ptxt_start, 
            data.key[0], data.value[0], data.key_ind[0], 
                key_rect_w, value_rect_w, data.key_seq, 
                    height, delay_time, speed)
                        .on("end", function () {
              d3.selectAll(".r_rows").style("opacity", 1)
              spread_anim_pivot(tbl_mid_xy, data.pivot[0], 
                data.pivot_rows, speed, delay_time, msg = true)
                .on("end", function () {
                  d3.selectAll(".removed").remove();
                  data.key_seq.forEach(function (d, i) {
                    current_chunk = all_o_rows.
                        slice(d["start"], d["stop"] + 1);
                    delay_time = spread_anim_move(tbl_mid_xy, 
                        current_chunk, data.key_rowseq[i],
                      data.key_ind[0], new_col_cnt, 
                        data.value_ind[0], speed, delay_time, 
                            msg);
                    msg = false;
                    new_col_cnt++;
                  })
                  spread_anim_fillna(data.na_pos, speed, 
                    delay_time);
                })
            });
        },
        resize: function (width, height) {}
      };
    }
  });

\end{lstlisting}

\subsection{\textbf{htmlwidgets} \textsf{JavaScript} code for \texttt{gather\_anim} \textbf{R}}

\begin{lstlisting}
HTMLWidgets.widget({
    name: 'gather_anim',
    type: 'output',
    factory: function (el, width, height) {
        let svg_width = width * 0.8;
        let svg_height = svg_width / 1.6;
        return {
            renderValue: function (x) {
                let data = x.data;
                svg_div = "animpanel0";
                let otbl_width = arr_sum(data.og_w[0]);
                let rtbl_width = arr_sum(data.rslt_w[0]);
                let or_width = otbl_width + rtbl_width;
                let cell_height = data.height[0];
                let xscale_num = (or_width + 1.5 * 
                    rtbl_width) / 0.9;
                let yscale_num = cell_height * 
                    (data.result.length +  data.result.length 
                        - 1) / 0.9;
                el = d3.select("body")
                    .append("div")
                    .attr("id", svg_div);
                el.append("svg")
                    .attr("width", svg_width)
                    .attr("height", svg_height);
                let x_scale =
                    d3.scaleLinear()
                    .domain([0, xscale_num])
                    .range([0, svg_width]);
                let y_scale =
                    d3.scaleLinear()
                    .domain([0, yscale_num])
                    .range([0, svg_height]);
                let height = y_scale(cell_height);
                let otbl_start = {
                    x: svg_width * 0.05,
                    y: svg_height * 0.25
                }
                let ptxt_start = {
                    x: svg_width * 0.015,
                    y: svg_height * 0.05
                }
                let o_width = arr_scale(data.og_w[0], 
                    x_scale);
                let x_cord_o = arr_scale(data.og_cord_x, 
                    x_scale);
                let rslt_cord_x = arr_scale(data.rslt_cord_x, 
                    x_scale);
                let r_width = arr_scale(data.rslt_w[0], 
                    x_scale);
                let r_tbl_tl = {
                    x: svg_width - svg_width * 0.05 - 
                        arr_sum(r_width),
                    y: otbl_start["y"]
                };
                let key_rect_w = x_scale(data.rslt_w[0]
                    [data.key_ind[0] - 1]);
                let value_rect_w = x_scale(data.rslt_w[0]
                    [data.value_ind[0] - 1]);
                let tbl_mid_xy = {
                    x: svg_width / 2,
                    y: svg_height / 2
                };
                draw_table_rectonly(data.result, 
                    r_tbl_tl["x"], r_tbl_tl["y"], rslt_cord_x,
                    r_width, height, "r", 
                        parseInt(data.pivot_ind));
                d3.selectAll(".r_rows")
                    .style("opacity", 0);
                input_text_node = d3.select(".r_rows").
                    selectAll(".r_cols").nodes();
                input_text_node.forEach(function (d, i) {
                    cur_rect = d3.select(input_text_node[i])
                        .select("rect");
                    d3.select(input_text_node[i])
                        .append("text")
                        .attr("x", parseFloat
                            (cur_rect.attr("x")) +     
                                parseFloat(cur_rect.
                                    attr("width") / 2)).attr("y", 
                            parseFloat(cur_rect.attr("y")) + 
                                parseFloat(cur_rect.
                                    attr("height") / 2))
                        .style("font-size", height * 0.5)
                        .text(data.rslt_cn[i]);
                })
                x_rect_cord = draw_table(data.original, 
                    otbl_start["x"], otbl_start["y"], x_cord_o,
                    o_width, height, "o", 
                        parseInt(data.pivot_ind));
                let all_o_rows = d3.selectAll(".o_rows")
                    .nodes();
                let all_r_rows = d3.selectAll(".r_rows")
                    .nodes();
                let speed = 1;
                let chunk_increment = all_o_rows.length - 1;
                let start_chunk = 1;
                let delay_time = 0;
                let tran_time = 2500 / speed;
                let msg = true;
                let first = true;
         prepare_gather(ptxt_start, data.col_ind, 
            data.key[0], data.value[0], key_rect_w, 
                value_rect_w, height, delay_time, 1)
            .on("end", function () {
                data.col_ind.forEach(function (d, i) {
                    end_chunk = start_chunk + 
                    chunk_increment;
                    current_chunk = all_r_rows.
                    slice(start_chunk, end_chunk)
                    delay_time = gather_anim_pivot
                    (tbl_mid_xy, current_chunk, 1, 1, 
                    speed, delay_time, msg, first);
                    delay_time = gather_anim_key
                        (tbl_mid_xy, current_chunk, 
                            data.col[i], d, data.key[0], 
                                data.key_ind[0], speed, 
                                    delay_time, msg);
                    delay_time = gather_anim_value
                        (tbl_mid_xy, current_chunk, data.col[i],d, 
                            data.value[0], data.value_ind[0], 
                                speed, delay_time, msg);
                    if (data.col.length - 1 != i) {
                        iso_msgbox(tbl_mid_xy["x"], 
                            tbl_mid_xy["y"], height * 15, 
                                height,`Now start moving 
                                    the information about 
                                        ${data.col[i + 1]} into 
                                    place `, delay_time, 
                                        tran_time,  0.5, 
                                            tran_time / 2);
                    delay_time = delay_time + tran_time;
                            }
                            msg = false;
                            first = false;
                            start_chunk = end_chunk;
                        })
                    });
            },
            resize: function (width, height) {}
        };
    }
});
\end{lstlisting}

\section{Helper function for both modules}

\begin{lstlisting}
include_cn = function(obj, ...){
  obj = as_tibble(apply(obj, 2, function(x) {
    if(is.numeric(x)) {
      x = round(x, 3)
    }
    return(x)
  }))
  return(rbind(colnames(obj), obj))
}
outersect = function(x, y){
  sort(c(setdiff(x, y),
         setdiff(y, x)))
}
col_max_width = function(obj, width, ...){
  obj = as_tibble(apply(obj, 2, as.character))
  obj = obj %>% transmute_all(nchar) %>%
    summarise_all(max, na.rm = T) %>% as.matrix()
  obj = ifelse(obj < width, width, obj)
  return(obj)
}
tbl_cord_x = function(x) {
  x = cumsum(x)
  return(c(0, x[-length(x)]))
}
is.integer0 <- function(x){
  is.integer(x) && length(x) == 0L
}
prepare_table = function(data, height = 3, width = 3) {
  data = include_cn(data)
  col_width = col_max_width(data)
  col_width = ifelse(col_width < width, width, col_width)
  return(list(data = data, col_width = col_width,
              height = height,
              cord_x = tbl_cord_x(col_width)))
}
num_colcharacter = function(df, key) {
  if(missing(key)) {
    stop("Must supply key")
  }
  df = df %>% select(-key)
  info = sapply(df, class)
  return(max(sum(info == "numeric"), sum(info == "factor")))
  }


\end{lstlisting}